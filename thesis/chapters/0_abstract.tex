
\section*{Executive Summary}
Rewards motivate people to complete actions. Habit formation systems use rewards to motivate people to form habits. This thesis looks at the effect of three types of positive reinforcement rewards on habit formation delivered by a chatbot, from three modes: visual, auditory and visual-auditory combined. The findings are evaluated against two hypotheses: i) rewards affect on habit performance and automaticity, ii) multiple modalities affect on habit performance and automaticity. 60 people participated in a 4-week study followed by voluntary semi-structured interviews. The findings showed that participants receiving the bot-delivered rewards had higher habit performance than the control group without rewards. A correlation was found between the habit formation method and habit automaticity. However, all participants interviewed (N = 7) found a drop in habit performance after one week without the prototype. Further research for using different rewards with behaviour change technology is needed to validate how each modality affected habit automaticity and habit performance.


This research presents four main contributions:

\begin{enumerate}
  \item A review of existing habit formation techniques, the impact of rewards from different modalities and how behaviour change interventions are used in technology.
  \item Design guidelines for building chatbots that deliver rewards from different modalities.
  \item The construction of Harry's Habits. A chatbot to track habits and deliver rewards from three modalities: visual, auditory and visual-auditory.
  \item Conducting and evaluating Harry's Habits during a 4-week user study with 60 participants to test the impact of each reward on habit automaticity and habit performance.
\end{enumerate}

\newpage



