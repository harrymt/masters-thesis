
\section*{Executive Summary}
Habit formation systems use rewards to motivate people to form habits. This thesis looks at the effect of three types of positive reinforcement rewards on habit formation. The rewards are delivered by a chatbot from three modes: visual, auditory and visual-auditory combined. The findings are evaluated against two hypotheses: i) positive reinforcement is effective at supporting habit formation by increasing automaticity and regular habit performance, ii) multiple modalities rewards are more effective then singular mode rewards. 60 people were recruited to participate in a 4-week study followed by voluntary semi-structured interviews. The findings showed that participants receiving the bot-delivered rewards had higher habit performance than the control group without rewards. Participants with visual-auditory rewards had the highest habit automaticity score, however, these were not statistically significant. Finally, all participants interviewed (N = 7) reported a drop in habit performance after one week without the prototype. Further research for using different rewards with behaviour change technology is needed to validate how each modality affected habit automaticity and habit performance.

This research presents four main contributions:

\begin{enumerate}
  \item A review of existing habit formation techniques, the impact of rewards from different modalities and how behaviour change interventions are used in technology.
  \item Design guidelines for building chatbots that deliver rewards based on different modalities.
  \item Technical specification and implementation details of Harry's Habits --- a chatbot to track habits and deliver rewards from three modalities: visual, auditory and visual-auditory.
  \item A description and results of a 4-week user study with 60 participants to test the impact of each reward on habit automaticity and habit performance.
\end{enumerate}

\newpage



