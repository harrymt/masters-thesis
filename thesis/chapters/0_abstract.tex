
\section*{Executive Summary}
Habit formation technologies use rewards and points as means for providing positive reinforcement, often in the format of visual or audio rewards such as jingles, badges or animations. Providing the right reward increases the chances for developing a new habit; yet, research on how these rewards should be delivered and the impact that this has on the process of habit formation is scarce. In this thesis we investigate how three types of positive reinforcement (visual, auditory, visual-auditory) influence habit performance and automaticity. Sixty people participated in a 4-week study in which they used a custom built chatbot that delivered different types of rewards for completing a new daily habit. The results show higher habit performance rates when a reward was present without necessarily increasing behaviour automaticity. This has implications for the design of habit formation technologies that rely on audio and visual rewards as means of positive reinforcement.

This research presents four main contributions:

\begin{enumerate}
  \item A review of existing habit formation techniques, the impact of rewards from different modalities and how behaviour change interventions are used in technology.
  \item Design guidelines for building chatbots that deliver rewards based on different modalities.
  \item Technical specification and implementation details of \textit{Harry's Habits} --- a chatbot to track habits and deliver rewards from three modalities: visual, auditory and visual-auditory.
  \item A description and results of a 4-week user study where sixty people participated to test the impact of each reward on habit automaticity and habit performance.
\end{enumerate}

\newpage
