
\newpage
\section{Evaluation}

Literature demonstrates the importance of designing for health and behaviour change~\cite{article_mhealth, article_designing_for_healthy_lifestyles, article_designing_for_health_behaviour_change_hci} but each have varying opinions about how to evaluate these systems. Klasnja et al.~\cite{article_evaluate_tech_health_behaviour_change} focuses on the usability of the system and if it meets users needs. However, Stawarz and Cox~\cite{article_designing_for_health_behaviour_change_hci} argue that when designing for health behaviour change, we must look into evaluating the system from other fields to properly consider the effectiveness of a system. The validated Behaviour Change Wheel Framework~\cite{article_behaviour_change_wheel} does just this, evaluating the system using validated behaviour change techniques from multiple domains. This project will use this framework to evaluate the system with evaluation trials testing the long-term effects and efficiency drawn from two fields of study, HCI research and health psychology.

\subsection{Evaluation trials}
The final part of the Behaviour Change Wheel Framework~\cite{article_behaviour_change_wheel} highlights and HCI research that focuses on health interventions~\cite{article_mhealth} demonstrates the importance of evaluation trials for evaluating mobile health systems. These trials have three goals to test: objective-quantitative efficacy, subjective-qualitative feedback measures and real-world feedback about how the system is utilised~\cite{article_evaluate_tech_health_behaviour_change}.\newline
\newline
The length of the trial will be based on two factors, the time needed to form a habit~\cite{article_how_habits_formed_modelling_habit_formation} and a previous habit formation trial~\cite{article_beyond_self_tracking_designing_apps}. First, the number of repetitive days required for an action to be considered a habit varies based on the complexity of the action~\cite{article_how_habits_formed_modelling_habit_formation}. Simple actions, such as drinking 2 glasses of water a day, can take a minimum of 18 days to form. The suggested actions used for this project will be considered as simple, e.g. stretching for 30 seconds. Second, a previous evaluation trial on habit-formation systems~\cite{article_how_habits_formed_modelling_habit_formation} showed an increase in habit automaticity after 4 weeks. This project will mirror this timeframe.\newline
\newline
A 4-week evaluation trial will test the success of the chatbot by evaluating the tool and the effectiveness of each modality on users habit strength. Chatbot interaction will be removed during the follow up study to test if users continue with the habit. Participants will split into four groups, all groups will receive reminders, three groups will receive rewards each from a different modality, and one group (control group) will not recieve any rewards.

\subsection{Testing Habit Strength}
Habit strength will be measured using a validated 12-question habit strength questionnaire that specifically looks at automaticity~\cite{article_habit_strength}. Automaticity will also be measured using a validated subset of the questionnaire from~\cite{article_habit_strength} to test users habit behavioural automaticity index~\cite{article_habit_measurement}. This will show the impact each modality has on habit automaticity and test the hypothesis. These questionnaires~\cite{article_habit_strength, article_habit_measurement} will occur half-way through the study (at 2-weeks), after the study has finished and after the follow up study.
