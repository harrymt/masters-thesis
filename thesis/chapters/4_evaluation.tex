
\newpage
\section{Evaluation}

Many people agree about the importance of designing systems for health and behaviour
change~\cite{article_mhealth, article_designing_for_healthy_lifestyles, article_designing_for_health_behaviour_change_hci}.
But each have varying opinions about how to evaluate these systems. Klasnja et al.~\cite{article_evaluate_tech_health_behaviour_change} focuses on system usability and does it meet the needs of users.
Whereas, Stawarz and Cox~\cite{article_designing_for_health_behaviour_change_hci} argue evaluating a system of this type requires information from other fields
to properly consider the systems effectiveness. The validated Behaviour Change Wheel Framework~\cite{article_behaviour_change_wheel} does just this.
Evaluating the system with validated behaviour change techniques from multiple domains. This project will use this framework to evaluate the chatbot with evaluation trials.
These will test the long-term effect and efficiency of the bot, with information from two fields of study, HCI and health psychology.

\subsection{Evaluation Trials}
Evaluation trials are the final part of the Behaviour Change Wheel Framework~\cite{article_behaviour_change_wheel}.
HCI research that focuses on health interventions~\cite{article_mhealth}, demonstrates the importance of evaluation trials for evaluating mobile health systems.
These trials have three goals to test: objective-quantitative efficacy, subjective-qualitative feedback measures and real-world feedback about how the system is
utilised~\cite{article_evaluate_tech_health_behaviour_change}. I will conduct an evaluation trial for this project.\newline
\newline
The length of the trial will be based on two factors, the time needed to form a habit~\cite{article_how_habits_formed_modelling_habit_formation} and the results of a previous
habit formation trial~\cite{article_beyond_self_tracking_designing_apps}.
First, the number of repetitive days required for an action to be considered a habit varies based on the complexity of the action~\cite{article_how_habits_formed_modelling_habit_formation}.
Simple actions, such as drinking 2 glasses of water a day, can take a minimum of 18 days to form.
The suggested actions used for this project will be considered as simple, e.g. stretching for 30 seconds.
Second, a previous evaluation trial on habit-formation systems~\cite{article_how_habits_formed_modelling_habit_formation} showed an increase in habit automaticity after 4 weeks.
This project will mirror that timeframe.\newline
\newline
A 4-week evaluation trial will test the success of the chatbot by evaluating the tool and the effectiveness of each modality on users habit strength.
Chatbot interaction will be removed during the follow up study to test if users continue with the habit.
Participants will split into four groups, all groups will receive reminders, three groups will receive rewards each from a different modality,
and one group (control group) will not recieve any rewards.

\subsection{Testing Habit Strength}
Habit strength will be measured using a validated 12-question questionnaire that specifically looks at automaticity~\cite{article_habit_strength}.
Automaticity will also be measured using a validated subset of the questionnaire from~\cite{article_habit_strength} to test users habit behavioural
automaticity index~\cite{article_habit_measurement}. This will show the impact each modality has on habit automaticity and test the hypothesis.
Participants will fill out the questionnaires~\cite{article_habit_strength, article_habit_measurement} at three stages: half-way through the trial (at 2-weeks),
after the trial has finished and after the follow up trial.
