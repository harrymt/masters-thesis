
\section*{Abstract}

Habits are automatic actions that require little effort. A simple action such as turning on the light to a room happens automatically, even if the light is already on. Forming new habits requires contextual cues, positive reinforcement and repetition. This research focuses on positive reinforcement in the form of rewards, experimenting with forming new habits using rewards from different modes. These rewards are delivered by a prototype chatbot, evaluated during a 28-day trial, testing the success of delivering rewards via a chatbot and the effectiveness of each modality on participants habit strength. 38 participants messaged a Facebook Messenger chatbot every day for 21 days, confirming or denying if they have performed their chosen habit (20 Press-Ups). After the evaluation trial, a 7-day follow up trial without interaction with the chatbot tests if participants continue performing their chosen habit. At the end of this 7-day period, participants evaluated the project success through an informal interview and a final validated habit formation questionnaire.

In summary this research:

\begin{enumerate}
  \item Compares how rewards from different modalities effect the building of new positive habits
  \item Evaluates a prototype chatbot using 28 real-users, revealing positive and negative aspects of the design.
  \item Investigates the use of chatbots as vehicles for promoting behaviour change.
\end{enumerate}
\newpage

\section*{Definitions}

\textbf{Human-Computer Interaction (HCI)} - Field of computer science that studies how people interact with computers.\newline
\newline
\textbf{Modality} - In the context of HCI, a modality or mode is the classification of a single independent channel of sensory input or output between a computer and a human.\newline
\newline
\textbf{System with Different Modalities} - A system that provides the user with multiple modes of interaction.\newline
\newline
\textbf{Chatbot} - A method of communicating with a computer system via a conversation using natural language.

\newpage
