\section*{Executive Summary}
Habits are automatic actions that require little effort. The action of washing your hands automatically after using the toilet is an example of a positive habit. Three elements are required to form a new positive habit: repetition, contextual cues and positive reinforcement. The contextual cue acts as a trigger and the positive reinforcement rewards the person, encouraging them to perform the action again when triggered by the cue. The strength of a habit can be measured using a validated questionnaire that specifically looks at how automatic the action is after the trigger occurs. Literature uses this habit strength measurement to show on average a habit takes 66 days to form \cite{article_how_habits_formed_modelling_habit_formation}. However, studies \cite{article_promoting_habit_formation, article_the_habitual_consumer} show people still fail at forming new positive habits and many give up because people don't stick to a routine.\newline
\newline
Mobile technology can help people stick to a routine for forming a habit that focuses on the three elements, by using reminders as a trigger to repeat an action, helping us associate contextual information around the action and rewarding us with positive reinforcement. However most existing mobile systems are not grounded in theory and build repetitive actions rather than habit automaticity. Therefore, people become dependent on technology rather than the habit and when the system is eventually removed, people's habit performance decreases.\newline
\newline
Building habit automaticity is the key to removing this dependency. Studies \cite{article_a_self_efficacy, article_meta_analytic_review_intrinsic_motivation} show that building motivation to complete the action increases habit automaticity and removes this dependency between the mobile technology and the user. We can encourage motivation by rewarding users with positive reinforcement by granting user's satisfaction after completing the action. However, what the reward is and how it is delivered is very important. Rewarding people with monetary gains (extrinsic rewards) hinders motivation, so rewards that benefit the person with satisfaction (intrinsic rewards) should be preferred. In addition, the type of reward should be delivered to suit each individual users needs, therefore a system should have a choice of reward delivery. Different research \cite{article_user_centred_multimodal_reminders} shows how interaction with users should span different modalities to suit the needs to users. This project will combine this knowledge from these two domains to focus on how intrinsic rewards from different modalities affect people's habit strength.\newline
\newline
This project will use the three elements of habit formation to build a mobile technology tool to deliver rewards from different modalities. These rewards will be delivered using a chatbot instead of a mobile app to present a novel method of interacting with users. Chatbots are a method of communicating with a computer system using natural language, providing deeper integration into users mobile phone and can hook into messaging services users are familiar with, such as Facebook Messenger. The chatbot will be built using the three elements of habit formation, providing a vehicle for repetition, contextual cues and positive reinforcement. The chatbot design will be based on a combined set of theory-based design requirements; one for building habit formation apps that aim to increase habit automaticity and one for increasing motivation by delivering intrinsic rewards.\newline
\newline
An evaluation trial will test the chatbot tool during a 4-week controlled period to measure user's habit strength. Participants will engage daily with the bot for 4 weeks, then for a further 1 week all interaction with the bot will be stopped to test if users continue to perform their chosen habit after the bot is removed. The chatbot will provide habit tracking using reminders as triggers, and rewards from three modalities, visual, auditory and tactile vibration. Participants will split into four groups, a control group that will not receive any rewards and the remaining three groups that will each receive rewards from a single different modality. The evaluation trial will test the success of the chatbot by measuring user's habit strength after the study, comparing the results with the control group.\newline
\newline
The project aims to deliver insight into how rewards from different modalities effect habit strength and opens up new research avenues for investigating the use of chatbots as vehicles for promoting behaviour change.
\newpage

\section*{Definitions}

\textbf{Human-Computer Interaction (HCI)} - Field of computer science that studies how people interact with computers.\newline
\newline
\textbf{Modality} - In the context of HCI, a modality or mode is the classification of a single independent channel of sensory input or output between a computer and a human.\newline
\newline
\textbf{Multimodal System} - A system that provides the user with multiple modes of interaction.\newline
\newline
\textbf{Chatbot} - A method of communicating with a computer system via a conversation using natural language.

\newpage
