\newpage

\section{Habit Formation}

TODO: Add `back-up notifications' instead of snooze, based on beyond tracking paper, 3 paragraphys before, STUDY 1 heading. Cite her research which answers the question *why* you need ot have back-up notifications as snoozing reminders, as routine changes.

To understand how to build a system that supports habit formation, we must discuss how people fundamentally form habits.\newline
\newline
Psychology defines habits as learned automatic cue-response actions, such actions that will perform automatically in response to another action or trigger that has been actioned repeatedly in the past~\cite{article_the_habitual_consumer}. Studies have shown people must keep to strict strategies and perform an action repeatedly before it turns into a action that occurs with little concious thought~\cite{article_promoting_habit_formation}.

\subsection{Three Elements of Habit Formation}
Forming a habit occurs similarly to how a person changes their behaviour. A study into how habits are formed~\cite{article_experiences_of_habit_formation} shows that using the following three elements ensures an action becomes permanent.

\begin{enumerate}
  \item Repetition
  \item Contextual Cues
  \item Positive Reinforcement
\end{enumerate}

\subsection*{Repetition}
Lally~et~al.~\cite{article_how_habits_formed_modelling_habit_formation} conducted a test on how long it takes for an action to become automatic, showing that the process of creating a new habit takes on average 66 days of repetitive use. The easier the action, the shorter the before the action turns into a habit, from drinking water (18 days), to going to the gym (254 days)~\cite{article_how_habits_formed_modelling_habit_formation}. However, applying the other two elements is needed before the action develops into a habit.

\subsection*{Contextual Cues}
Context from information around the action, serves as a cue to trigger events to push the person onto performing the action. For example, if you wanted to adopt a stretching habit, you could attach it onto an existing habit like brushing your teeth. The contextual cue of brushing your teeth will trigger you to stretch.\newline
\newline
Behaviour change literature~\cite{article_implementation_intentions_multicue} shows that attaching habits onto existing event-based cues are easier to remember, when compared with time-based habits, e.g. stretch every 4 hours. These help connect the contextual information with the action and builds habit automaticity~\cite{article_implementation_intentions}. Further research into the design implications of contextual cues shows how multi-cue routines are more effective that a single cue~\cite{article_understanding_use_contextual_cues_design_impl}.

\subsection*{Positive Reinforcement}
Self-efficacy, the belief in ones ability to succeed, plays a large part in forming habits and is a main part of behaviour change~\cite{article_a_self_efficacy}. Rewards give people this experience by feeding back their success of their action. Rewarding a person with positive reinforcement, strengthens the habit by giving the feeling of satisfaction~\cite{article_promoting_habit_formation}. However, the type of reward matters, as rewards that benefit the person with satisfaction (intrinsic rewards) should be used over monetary gains (extrinsic rewards), due to issues with extrinsic rewards hindering motivation~\cite{article_meta_analytic_review_intrinsic_motivation}.

\subsection{Technology}
Research~\cite{survey_on_apps_2,survey_on_current_apps_of_steel} into how mobile systems can support habit formation and behaviour change, shows a large number of habit forming systems are mobile apps. Studies into the effectiveness of these apps has been recently conducted~\cite{article_beyond_self_tracking_designing_apps, article_dont_kick_habit} revealing that although most of these apps are rated highly, they do not ground themselves in behaviour change theory. Further surveys of these apps~\cite{survey_on_current_apps_of_steel,survey_on_apps_2} suggest that habit performance is not sustained when the app is removed, due to the lack of habit automaticity built during the habit tracking process. Using apps consistently to manage behaviour can create a notable difference in the person when the system is removed~\cite{article_my_phone_is_part_of_my_soul}. This is also the case with many behaviour change systems, when the system is removed any improved performance is lost~\cite{article_dont_kick_habit, article_realtime_feedback_improving_medication_taking}.


To build automaticity with technology, two strategies for building habit formation systems, that ground themselves in theory, are reviewed: Stawarz et al.~\cite{article_beyond_self_tracking_designing_apps} discuss guidelines for building habit forming systems and Weiser et al.~\cite{article_taxonomy_motivational_affordances_meaningful} show that `\textit{motivation is a key requirement for behaviour change}' presenting five design principles and six requirements about the implementation mechanics of habit forming systems that focus on rewards and motivational needs. These share three recommendations: i) make personalized, well-defined, structured multi-cue routines, with examples and support users choice of not setting remembering strategies, ii) reminders can effectively support prospective memory in the short-term, increasing the logging of health data~\cite{the_power_of_logging_mobile_notifications} and educating them about how they should perform in the long-term, iii) rewards are a good form of external motivation because they don't change the ability to perform a behaviour, unless the reward itself is a tool that increases ability. These rewards provide a strong motivational source, but like all extrinsic motivators, these are less effective for changing behaviour in the long run, because externally motivated behaviour lasts as long as the external motivator exists. This project builds upon this set of requirements, combining them into a new set of design requirements for habit formation systems that focuses on rewards.

\subsection{Combined Requirements}
Combining information from the above two sources~\cite{thesis_kathy, article_taxonomy_motivational_affordances_meaningful}, produces the following list of methods to build a habit forming app.\newline
\newline
The system must:

\begin{enumerate}
    \item Help users define a memorable strategy.
    \item Give users small difficult tasks.
    \item Enable competition \& comparison \& cooperation.
    \item Show insights for improvements and support changes.
    \item Remind them about cues and remembering strategies.
    \item Reward users.
    \item Disable cue reminders when behaviour is routine.
    \item Check if the action has already happened.
\end{enumerate}

The following table shows how the two sources~\cite{thesis_kathy, article_taxonomy_motivational_affordances_meaningful} were combined.

\begin{table}
  \centering
  \begin{tabular}{l | l}
    {\small\textit{\#}}
    & {\small \textit{Requirement}}\\
    \midrule
    1 & Help users define a memorable strategy. \\
    2 & Give users small difficult tasks. \\
    3 & Enable competition \& comparison \& cooperation. \\
    4 & Show insights for improvements and support changes. \\
    5 & Remind them about cues and remembering strategies. \\
    6 & Reward users. \\
    7 & Disable cue reminders when behaviour is routine. \\
    8 & Check if the action has already happened. \\
  \end{tabular}
  \caption{Requirements for building mobile systems that focus on rewards.}~\label{tab:requirements_mine}
\end{table}


\begin{landscape}
% \newgeometry{top=2.3cm, bottom=2.4cm, left=1.9cm, right=2cm} % abstract margins
\renewcommand{\arraystretch}{1.5} % Increase line height of the following tables
\begin{figure}[ht] % ht
    % \centering


\begin{center}
\begin{tabular}{ |p{.9cm}|p{6.5cm}|p{6.5cm}|p{5.8cm}| }
  \hline
  \textbf{REQ} & \textbf{Stawarz.~\cite{thesis_kathy}} & \textbf{Weiser et al.~\cite{article_taxonomy_motivational_affordances_meaningful}} & \textbf{Combined Requirement} \\ \hline % Row
  1.  & REQ 1. Help users define a good remembering strategy.\newline REQ 2. Provide examples of good remembering strategies & Design REQ 2. Support User Choice.\newline Design REQ 4. Provide personalized experience. & Help users define a memorable strategy. \\ \hline
  1a. & N\/A & Mechanic REQ 3. Offer challenges & Give users small difficult tasks (challenges). \\ \hline
  1b. & N\/A & Mechanic REQ 5. Competition \& Comparison.\newline Mechanic REQ 6. Cooperation & Give them Competition \& Comparison \& Cooperation. \\ \hline
  2.  & REQ 3. Provide suggestions for strategy improvements and support changes. & Design REQ 1. Offer meaningful suggestions.\newline Mechanic REQ 2. User Education. & Give insights for strategy improvements and support changes. \\ \hline
  3.  & REQ 4. Remind about cues and remembering strategies. & Design REQ 3. Provide User Guidance.\newline Mechanic REQ 2. User Education. & Remind them about cues and remembering strategies. \\ \hline
  4.  & N\/A & Mechanic REQ 4. Rewards. & Reward users. \\ \hline
  5.  & REQ 5. Disable cue reminders when the behaviour becomes a part of a routine. & Design REQ 5. Design for every stage of behaviour change. & Disable cue reminders when behaviour is routine. \\ \hline
  6.  & REQ 6. Help users check whether the habit has already happened. & Mechanic REQ 1. Feedback. & Check if the action has already happened. \\ \hline
\end{tabular}
\end{center}

    \caption{The combined table of requirements for designing mobile systems focused on rewards.}
    \label{fig:reqtable}
\end{figure}
\newgeometry{top=2.4cm, bottom=2.5cm, left=2cm, right=2cm}
\newpage
\end{landscape}

\subsection{Requirements Overview}

The combined requirements based on methods from~\cite{thesis_kathy, article_taxonomy_motivational_affordances_meaningful} create a list grounded in habit formation theory and focused on rewards. Each requirement provides detailed breakdown about why it's used and what mechanics it relates to from theory.


\subsection{Requirements}
\subsubsection{1. Help users define a memorable strategy.}
Make personalized, well defined, structured multi-cue routines \& support users choice of not setting remembering strategies. In addition, provide examples of some strategies to users.

\subsubsection*{2. Give users small difficult tasks.}
Turn the bigger habit into smaller assignments to make it more enjoyable, being careful to not make them forced. These should be user specified to support user autonomy, but should be specific and challenging to get better results.

\subsubsection*{3. Enable competition \& comparison \& cooperation.}
Friends, teams, groups, leader-boards and collections.

\subsubsection*{4. Show insights for improvements and support changes.}
Give users meaningful \& accumulated  instant feedback based on their system usage.

\subsubsection*{5. Remind them about cues and remembering strategies.}
Reminders can effectively support prospective memory in the short-term, increase the logging of health data~\cite{the_power_of_logging_mobile_notifications} and educate them about what they should perform in the long-term.

\subsubsection*{6. Reward users.}
Rewards are a good form of external motivation because they don't change the ability to perform a behaviour, unless the reward itself is a tool that increases ability. These rewards provide a strong motivational source, but like all extrinsic motivators, these are less effective for changing behaviour in the long run, because externally motivated behaviour lasts as long as the external motivator exists. Use achievements and badges as means to identify methods that enable internalisation of externally motivated behaviour.

\subsubsection*{7. Disable cue reminders when behaviour is routine.}
Relying on reminders in the long-term can hinder habit development, therefore ease off from reminders later.

\subsubsection*{8. Check if the action has already happened.}
People find it easy to forget whether an automatic task was completed, check to see if the action has already happened.



\subsection*{Summary}

These requirements act as the basis for this project. They are guidelines for building habit formation systems that focus on delivering rewards to build habit automaticity and motivation. This project tests how different types of rewards from different modalities effect users habit strength.
