\newpage

\section{Background}

\subsection{Habit Formation}
Habits are automatic actions that require little concious effort~\cite{article_the_habitual_consumer}. To develop new habits people must keep to a strict strategy and perform the action repeatedly to strengthen the automaticity of the action~\cite{article_promoting_habit_formation}. When strong habits are developed the likelihood of behaviours persisting is higher~\cite{putting_habit_into_practice} and habits are more effectively developed when specific and measurable goals are set~\cite{habits_better_when_have_specific_and_measurable_goals}. Changing behaviour requires the formation of new habits to make this change permanent. Three elements are needed to form a habit: positive reinforcement, repetition and contextual cues~\cite{article_experiences_of_habit_formation}. Positive reinforcement rewards the person by encouraging them to perform the action again until it forms into a habit which significantly increases intrinsic motivation and increases the persons perception of their own performance~\cite{positive_reinforcement_pro}. Contextual cues act as triggers with constant repetition and dependning on the complexity of the behaviour, the time required to form a new habit varies from 18 to 254 days~\cite{article_how_habits_formed_modelling_habit_formation}.
However, people still fail at forming new positive habits and give up, often due to their lack of routine~\cite{article_the_habitual_consumer, article_promoting_habit_formation}.

\subsubsection*{Contextual Cues}
Context from information around the action, serves as a cue to trigger events to push the person onto performing the action. For example, if you wanted to adopt a habit to floss your teeth, you could attach it onto an existing habit like brushing your teeth. The contextual cue of brushing your teeth will trigger you to floss them. Behaviour change literature~\cite{article_implementation_intentions_multicue} shows that attaching habits onto existing event-based cues are easier to remember, when compared with time-based habits, e.g. take a break from your computer screen every \verb|x| hours, mainly due to changing environments~\cite{coaching_not_that_good}. These help connect the contextual information with the action and builds habit automaticity~\cite{article_implementation_intentions}. Further research into the design implications of contextual cues shows how multi-cue routines are more effective that a single cue~\cite{article_understanding_use_contextual_cues_design_impl}.

\subsubsection*{Repetition}
The process of creating a new habit takes on average 66 days of repetitive use~\cite{article_how_habits_formed_modelling_habit_formation}. The easier the action, the shorter time before the action turns into a habit, from drinking water (18 days), to going to the gym (254 days). However, existing routines and cues are needed before the action develops into a habit~\cite{habits_event_cues_1, habits_event_cues_2}. An existing routine acts as a trigger to motivate the desired action. Context from that routine serves as the cue for the trigger.

\subsubsection*{Rewards}
Rewards give motivation, fuelling the belief in success and self-efficacy, which plays a large part in forming habits. Some researchers~\cite{article_a_self_efficacy} suggest it is the main part of behaviour change. Variable types of rewards have been shown to increase dopamine in a laboratory study on rats~\cite{variable_rewards_increases_dopamine}. This technique has proved to be an effective method of increasing repetition as shown in slot machines~\cite{programme_why_are_gambling_machines_addictive} that vary their reward payout. But although variable rewards increase habit repetition, they hinder habit automaticity~\cite{variable_rewards_increases_dopamine}, which is key to creating permanent and long-lasting behaviour change. Rewards are a good form of external motivation because they don't change the ability to perform a behaviour, unless the reward itself is a tool that increases ability~\cite{article_taxonomy_motivational_affordances_meaningful}. Rewards provide a strong motivational source, but like all extrinsic motivators, these are less effective for changing behaviour in the long run, because externally motivated behaviour lasts as long as the external motivator exists~\cite{article_beyond_self_tracking_designing_apps}. One type of reward is positive reinforcement.

\subsection{Positive Reinforcement}
Positive reinforcement rewards the person by encouraging them to perform the action again until it forms into a habit. This reward increases intrinsic motivation by giving the feeling of satisfaction~\cite{article_promoting_habit_formation} and increases the persons perception of their performance~\cite{positive_reinforcement_pro} especially when the task is interesting~\cite{article_meta_analytic_review_intrinsic_motivation}. TTechnology uses different modalities to deliver positive reinforcement.


