\newpage

\section{Background}

\subsection{Habit Formation}
Habits are automatic actions that require little concious effort~\cite{article_the_habitual_consumer}. To develop new habits people must keep to a strict strategy and perform the action repeatedly to strengthen the automaticity of the action~\cite{article_promoting_habit_formation}. When strong habits are developed the likelihood of behaviours persisting is higher~\cite{putting_habit_into_practice} and habits are more effectively developed when specific and measurable goals are set~\cite{habits_better_when_have_specific_and_measurable_goals}. The formation of new habits requires behaviour change. Three elements are needed to make this change permanent: positive reinforcement, repetition and contextual cues~\cite{article_experiences_of_habit_formation}. Positive reinforcement rewards the person by encouraging them to perform the action again until it forms into a habit which significantly increases intrinsic motivation and increases the persons perception of their own performance~\cite{positive_reinforcement_pro}. Contextual cues act as triggers with constant repetition, as habits, on average, take up to 66 days to form~\cite{article_how_habits_formed_modelling_habit_formation}.
However, people still fail at forming new positive habits and give up, often due to their lack of routine~\cite{article_the_habitual_consumer, article_promoting_habit_formation}.

\subsubsection*{Contextual Cues}
Context from information around the action, serves as a cue to trigger events to push the person onto performing the action. For example, if you wanted to adopt a stretching habit, you could attach it onto an existing habit like brushing your teeth. The contextual cue of brushing your teeth will trigger you to stretch. Behaviour change literature~\cite{article_implementation_intentions_multicue} shows that attaching habits onto existing event-based cues are easier to remember, when compared with time-based habits, e.g. stretch every 4 hours. These help connect the contextual information with the action and builds habit automaticity~\cite{article_implementation_intentions}. Further research into the design implications of contextual cues shows how multi-cue routines are more effective that a single cue~\cite{article_understanding_use_contextual_cues_design_impl}.


\subsubsection*{Repetition}
The process of creating a new habit takes on average 66 days of repetitive use~\cite{article_how_habits_formed_modelling_habit_formation}. The easier the action, the shorter time before the action turns into a habit, from drinking water (18 days), to going to the gym (254 days). However, existing routines and cues are needed before the action develops into a habit~\cite{habits_event_cues_1, habits_event_cues_2}. An existing routine acts as a trigger to motivate the desired action. Context from that routine serves as the cue for the trigger. For example, if you wanted to adopt a habit to weigh yourself every day, you could attach it onto an existing habit like brushing your teeth. The contextual cue of brushing your teeth will trigger you to weigh yourself. When designing behaviour change interventions, using different types of cues can be beneficial. Multi-cue routines have shown to be more effective than a single cue~\cite{article_understanding_use_contextual_cues_design_impl}. Attaching habits onto existing event-based cues are easier to remember~\cite{article_implementation_intentions_multicue} when compared with time-based habits, mainly due to change in time with change in environment, e.g. the weekend~\cite{coaching_not_that_good}. Event-based cues help connect the contextual information with the action and builds habit automaticity~\cite{article_implementation_intentions}.

\subsubsection*{Rewards}
Rewards grant motivation, fuelling the belief in success and self-efficacy, which plays a large part in forming habits. Some researchers~\cite{article_a_self_efficacy} suggest it is the main part of behaviour change. Variable types of rewards have been shown to increase dopamine in a laboratory study on rats~\cite{variable_rewards_increases_dopamine}. This technique has proved to be an effect method of increasing repetition as shown in slot machines that vary their reward payout. But although variable rewards increase habit repetition, they hinder habit automaticity~\cite{variable_rewards_increases_dopamine}, which is the key to create permanent and long-lasting behaviour change.

\subsubsection*{Positive Reinforcement}
This thesis discusses intrinsic positive reinforcement as the method of reward, rather than other types of rewards e.g. negative reinforcement. Rewarding a person with intrinsic positive reinforcement strengthens the habit by giving the feeling of satisfaction~\cite{article_promoting_habit_formation}, particularly in relation to interesting tasks~\cite{article_meta_analytic_review_intrinsic_motivation}. This study explores different types of positive reinforcement rewards and how they effect behaviour change.


\subsubsection*{Technology}
Research~\cite{survey_on_apps_2,survey_on_current_apps_of_steel} into how mobile systems can support habit formation and behaviour change, shows a large number of habit forming systems are mobile apps. Studies into the effectiveness of these apps has been recently conducted~\cite{article_beyond_self_tracking_designing_apps, article_dont_kick_habit} revealing that although most of these apps are rated highly, they do not ground themselves in behaviour change theory. Further surveys of these apps~\cite{survey_on_current_apps_of_steel,survey_on_apps_2} suggest that habit performance is not sustained when the app is removed, due to the lack of habit automaticity built during the habit tracking process. Using apps consistently to manage behaviour can create a notable difference in the person when the system is removed~\cite{article_my_phone_is_part_of_my_soul}. This is also the case with many behaviour change systems, when the system is removed any improved performance is lost~\cite{article_dont_kick_habit, article_realtime_feedback_improving_medication_taking}.


Building habit automaticity requires the desired action to be built around an existing routine~\cite{article_how_habits_formed_modelling_habit_formation, article_implementation_intentions_multicue}. Technology should allow for this and help with routine creation~\cite{article_dont_forget_your_pill}. Additional checks should guard against changes in routine to remind people about their habit if their situation changes. Technology can be used to send back-up notifications and post-completion notifications to see if the action has already happened. This is vital for sustaining performance and building habit automaticity.


Two strategies for designing habit formation systems are reviewed: Stawarz et al.~\cite{article_beyond_self_tracking_designing_apps} discuss guidelines for building habit forming systems and Weiser et al.~\cite{article_taxonomy_motivational_affordances_meaningful} show that `\textit{motivation is a key requirement for behaviour change}' presenting five design principles and six requirements about the implementation mechanics of habit forming systems that focus on rewards and motivational needs. These share three recommendations: i) make personalized, well-defined, structured multi-cue routines, with examples and support users choice of not setting remembering strategies, ii) reminders can effectively support prospective memory in the short-term, increasing the logging of health data~\cite{the_power_of_logging_mobile_notifications} and educating them about how they should perform in the long-term, iii) rewards are a good form of external motivation because they don't change the ability to perform a behaviour, unless the reward itself is a tool that increases ability. These rewards provide a strong motivational source, but like all extrinsic motivators, these are less effective for changing behaviour in the long run, because externally motivated behaviour lasts as long as the external motivator exists. This project builds upon this set of requirements, combining them into a new set of design requirements for habit formation systems that focuses on rewards.

\subsection{Guidelines} \label{recommendations}
The combined guidelines based on methods from~\cite{thesis_kathy, article_taxonomy_motivational_affordances_meaningful} create a list grounded in habit formation theory and focused on rewards. Each requirement provides detailed breakdown about why it's used and what mechanics it relates to from theory.


\textbf{1. Help users define a memorable strategy.}\newline
Make personalized, well defined, structured multi-cue routines and support users choice of not setting remembering strategies. In addition, provide examples of some strategies to users.


\textbf{2. Give users small difficult tasks.}\newline
Turn the bigger habit into smaller assignments to make it more enjoyable, being careful to not make them forced. These should be user specified to support user autonomy, but should be specific and challenging to get better results.


\textbf{3. Enable competition, comparison and cooperation.}\newline
Friends, teams, groups, leader-boards and collections.


\textbf{4. Show insights for improvements and support changes.}\newline
Give users meaningful and accumulated instant feedback based on their system usage.


\textbf{5. Remind them about cues and remembering strategies.}\newline
Reminders can effectively support prospective memory in the short-term, increase the logging of health data~\cite{the_power_of_logging_mobile_notifications} and educate them about what they should perform in the long-term.


\textbf{6. Reward users.}\newline
Rewards are a good form of external motivation because they don't change the ability to perform a behaviour, unless the reward itself is a tool that increases ability. These rewards provide a strong motivational source, but like all extrinsic motivators, these are less effective for changing behaviour in the long run, because externally motivated behaviour lasts as long as the external motivator exists. Use achievements and badges as means to identify methods that enable internalisation of externally motivated behaviour.


\textbf{7. Disable cue reminders when behaviour is routine.}\newline
Relying on reminders in the long-term can hinder habit development, therefore ease off from reminders later.


\textbf{8. Check if the action has already happened.}\newline
People find it easy to forget whether an automatic task was completed, check to see if the action has already happened.


Using these recommendations for building habit formation systems that focus on delivering rewards, we decide what type of rewards are needed and discuss what modality would best suit positive reinforcement.

\begin{landscape}
% \newgeometry{top=2.3cm, bottom=2.4cm, left=1.9cm, right=2cm} % abstract margins
\renewcommand{\arraystretch}{1.5} % Increase line height of the following tables
\begin{figure}[ht] % ht
    % \centering


\begin{center}
\begin{tabular}{ |p{7cm}|p{6cm}|p{6.5cm}| }
  \hline
  \textbf{Combined} & \textbf{Stawarz.~\cite{thesis_kathy}} & \textbf{Weiser et al.~\cite{article_taxonomy_motivational_affordances_meaningful}} \\ \hline % Row
  1. Help users define a memorable strategy. & REQ 1. Help users define a good remembering strategy.\newline REQ 2. Provide examples of good remembering strategies & Design REQ 2. Support User Choice.\newline Design REQ 4. Provide personalized experience. \\ \hline
  2. Give users small difficult tasks & N\/A & Mechanic REQ 3. Offer challenges\\ \hline
  3. Enable competition, comparison and cooperation. & N\/A & Mechanic REQ 5. Competition \& Comparison.\newline Mechanic REQ 6. Cooperation\\ \hline
  4. Show insights for improvements and support changes. & REQ 3. Provide suggestions for strategy improvements and support changes. & Design REQ 1. Offer meaningful suggestions.\newline Mechanic REQ 2. User Education. \\ \hline
  5. Remind them about cues and remembering strategies. & REQ 4. Remind about cues and remembering strategies. & Design REQ 3. Provide User Guidance.\newline Mechanic REQ 2. User Education. \\ \hline
  6. Reward users. & N\/A & Mechanic REQ 4. Rewards. \\ \hline
  7. Disable cue reminders when behaviour is routine. & REQ 5. Disable cue reminders when the behaviour becomes a part of a routine. & Design REQ 5. Design for every stage of behaviour change. \\ \hline
  8. Check if the action has already happened. & REQ 6. Help users check whether the habit has already happened. & Mechanic REQ 1. Feedback. \\ \hline
\end{tabular}
\end{center}

    \caption{Guidelines for building habit formation systems that focus on rewards.}
    \label{fig:reqtable}
\end{figure}
\newgeometry{top=2.4cm, bottom=2.5cm, left=2cm, right=2cm}
\newpage
\end{landscape}
