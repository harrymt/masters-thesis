\newpage

\section{Habit Formation}

TODO: Add snooze to requirements, look into Kathy research, find research about *why* I should have snoozable reminders


To understand how to build a system that supports habit formation, we must discuss how people fundamentally form habits.\newline
\newline
Psychology defines habits as learned automatic cue-response actions, such actions that will perform automatically in response to another action or trigger that has been actioned repeatedly in the past~\cite{article_the_habitual_consumer}. Studies have shown people must keep to strict strategies and perform an action repeatedly before it turns into a action that occurs with little concious thought~\cite{article_promoting_habit_formation}.

\subsection{Three Elements of Habit Formation}
Forming a habit occurs similarly to how a person changes their behaviour. A study into how habits are formed~\cite{article_experiences_of_habit_formation} shows that using the following three elements ensures an action becomes permanent.

\begin{enumerate}
  \item Repetition
  \item Contextual Cues
  \item Positive Reinforcement
\end{enumerate}

\subsection*{Repetition}
Lally~et~al.~\cite{article_how_habits_formed_modelling_habit_formation} conducted a test on how long it takes for an action to become automatic, showing that the process of creating a new habit takes on average 66 days of repetitive use. The easier the action, the shorter the before the action turns into a habit, from drinking water (18 days), to going to the gym (254 days)~\cite{article_how_habits_formed_modelling_habit_formation}. However, applying the other two elements is needed before the action develops into a habit.

\subsection*{Contextual Cues}
Context from information around the action, serves as a cue to trigger events to push the person onto performing the action. For example, if you wanted to adopt a stretching habit, you could attach it onto an existing habit like brushing your teeth. The contextual cue of brushing your teeth will trigger you to stretch.\newline
\newline
Behaviour change literature~\cite{article_implementation_intentions_multicue} shows that attaching habits onto existing event-based cues are easier to remember, when compared with time-based habits, e.g. stretch every 4 hours. These help connect the contextual information with the action and builds habit automaticity~\cite{article_implementation_intentions}. Further research into the design implications of contextual cues shows how multi-cue routines are more effective that a single cue~\cite{article_understanding_use_contextual_cues_design_impl}.

\subsection*{Positive Reinforcement}
Self-efficacy, the belief in ones ability to succeed, plays a large part in forming habits and is a main part of behaviour change~\cite{article_a_self_efficacy}. Rewards give people this experience by feeding back their success of their action. Rewarding a person with positive reinforcement, strengthens the habit by giving the feeling of satisfaction~\cite{article_promoting_habit_formation}. However, the type of reward matters, as rewards that benefit the person with satisfaction (intrinsic rewards) should be used over monetary gains (extrinsic rewards), due to issues with extrinsic rewards hindering motivation~\cite{article_meta_analytic_review_intrinsic_motivation}.

\subsection{Technology}
Research into how mobile systems can support habit formation and behaviour change, shows a large number of habit forming systems are mobile apps. Studies into the effectiveness of these apps has been recently conducted~\cite{article_beyond_self_tracking_designing_apps, article_dont_kick_habit} revealing that although most of these apps are rated highly, they do not ground themselves in behaviour change theory. Research into some of these apps show that habits are not sustained when the app is removed, due to the lack of habit automaticity~\cite{article_beyond_self_tracking_designing_apps}.\newline
\newline
Two pieces of literature discuss different concrete strategies for building habit forming mobile apps that do ground themselves in theory~\cite{thesis_kathy, article_taxonomy_motivational_affordances_meaningful}. Stawarz~\cite{thesis_kathy} presented formal requirements for building habit forming apps, based on the above three elements of habit formation with the aim to build habit automaticity. Weiser et al.~\cite{article_taxonomy_motivational_affordances_meaningful} shows that `motivation is a key requirement for behaviour change' presenting five design principle requirements and six requirements about the implementation mechanics of habit forming systems that focus on rewards and motivational needs. This project will build upon this set of requirements, combing them into a new set of design requirements for habit formation systems that focuses on rewards.

\subsection{Combined Requirements}
Combining information from the above two sources~\cite{thesis_kathy, article_taxonomy_motivational_affordances_meaningful}, produces the following list of methods to build a habit forming app.\newline
\newline
The system must:

\begin{enumerate}
  \item Help users define a memorable strategy
  \item Give users small difficult tasks (challenges)
  \item Give them Competition \& Comparison \& Cooperation
  \item Give insights for strategy improvements and support changes
  \item Remind them about cues and remembering strategies
  \item Reward Users
  \item Disable cue reminders when behaviour is routine
  \item Check if the action has already happened
\end{enumerate}

The following table shows how the two sources~\cite{thesis_kathy, article_taxonomy_motivational_affordances_meaningful} were combined.

\begin{landscape}
% \newgeometry{top=2.3cm, bottom=2.4cm, left=1.9cm, right=2cm} % abstract margins
\renewcommand{\arraystretch}{1.5} % Increase line height of the following tables
\begin{figure}[ht] % ht
    % \centering


\begin{center}
\begin{tabular}{ |p{.9cm}|p{6.5cm}|p{6.5cm}|p{5.8cm}| }
  \hline
  \textbf{REQ} & \textbf{Stawarz.~\cite{thesis_kathy}} & \textbf{Weiser et al.~\cite{article_taxonomy_motivational_affordances_meaningful}} & \textbf{Combined Requirement} \\ \hline % Row
  1.  & REQ 1. Help users define a good remembering strategy.\newline REQ 2. Provide examples of good remembering strategies & Design REQ 2. Support User Choice.\newline Design REQ 4. Provide personalized experience. & Help users define a memorable strategy. \\ \hline
  1a. & N\/A & Mechanic REQ 3. Offer challenges & Give users small difficult tasks (challenges). \\ \hline
  1b. & N\/A & Mechanic REQ 5. Competition \& Comparison.\newline Mechanic REQ 6. Cooperation & Give them Competition \& Comparison \& Cooperation. \\ \hline
  2.  & REQ 3. Provide suggestions for strategy improvements and support changes. & Design REQ 1. Offer meaningful suggestions.\newline Mechanic REQ 2. User Education. & Give insights for strategy improvements and support changes. \\ \hline
  3.  & REQ 4. Remind about cues and remembering strategies. & Design REQ 3. Provide User Guidance.\newline Mechanic REQ 2. User Education. & Remind them about cues and remembering strategies. \\ \hline
  4.  & N\/A & Mechanic REQ 4. Rewards. & Reward users. \\ \hline
  5.  & REQ 5. Disable cue reminders when the behaviour becomes a part of a routine. & Design REQ 5. Design for every stage of behaviour change. & Disable cue reminders when behaviour is routine. \\ \hline
  6.  & REQ 6. Help users check whether the habit has already happened. & Mechanic REQ 1. Feedback. & Check if the action has already happened. \\ \hline
\end{tabular}
\end{center}

    \caption{The combined table of requirements for designing mobile systems focused on rewards.}
    \label{fig:reqtable}
\end{figure}
\newgeometry{top=2.4cm, bottom=2.5cm, left=2cm, right=2cm}
\newpage
\end{landscape}

\subsection{Requirements Overview}

The combined requirements based on methods from~\cite{thesis_kathy, article_taxonomy_motivational_affordances_meaningful} create a list grounded in habit formation theory and focused on rewards. Each requirement provides detailed breakdown about why it's used and what mechanics it relates to from theory.

\subsubsection*{1. Help users define a memorable strategy}
  \begin{itemize}
    \item Make personalized, well defined, structured multi-cue routines \& also support users choice of not setting remembering strategies.
    \item Provide examples of some strategies to users.
  \end{itemize}

\subsubsection*{1a. Give users small difficult tasks (challenges)}
  \begin{itemize}
    \item Assignments: Turn the bigger habit into smaller assignments to make it more enjoyable, being careful to not make them forced.
    \item Quests: Same as assignments, but optional.
    \item Goals: User specified to support user autonomy. Should be specific and challenging to get better results.
  \end{itemize}

\subsubsection*{1b. Give them Competition \& Comparison \& Cooperation}
  \begin{itemize}
    \item Friends, teams \& groups.
    \item Leader-boards and collections.
  \end{itemize}

\subsubsection*{2. Give insights for strategy improvements and support changes}
  \begin{itemize}
    \item Give users meaningful instant feedback \& accumulated feedback based on their system usage.
  \end{itemize}

\subsubsection*{3. Remind them about cues and remembering strategies}
  \begin{itemize}
    \item Reminders can effectively support prospective memory in the short term.
    \item Educate them about what they should perform.
  \end{itemize}

\subsubsection*{4. Rewards}
  \begin{itemize}
    \item A good form of external motivation because they don't change the ability to perform a behaviour, unless the reward itself is a tool that increases ability.
    \item Provide a strong motivational source, but like all extrinsic motivators, these are less effective for changing behaviour in the long run, because externally motivated behaviour lasts as long as the external motivator exists.
    \item Identifying methods that enable internalisation of externally motivated behaviour.
    \item Achievements and badges.
  \end{itemize}

\subsubsection*{5. Disable cue reminders when behaviour is routine}
  \begin{itemize}
    \item Relying on reminders in the long term can hinder habit development.
    \item Ease off from reminders later.
  \end{itemize}

\subsubsection*{6. Check if the action has already happened}
  \begin{itemize}
    \item Easy to forget whether a task done automatically, completed.
  \end{itemize}


\subsection*{Summary}

These requirements will act as the basis for this project. The are guidelines for building habit formation systems that focus on delivering rewards to build habit automaticity and motivation. This project aims to test how different types of rewards from different modalities effects users habit strength.
