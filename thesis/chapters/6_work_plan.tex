
\newpage
\section{Work Plan}
For this project there are four milestones that split into two parts, building and testing.


\subsection{Milestones}

\begin{enumerate}
  \item Building
  \begin{itemize}
    \item 1.1 Build the chatbot.
    \item 1.2 Build delivery of rewards from three modalities, visual, auditory and tactile vibration.
  \end{itemize}
  \item Testing
  \begin{itemize}
    \item 2.1 Conduct 4-week evaluation trial.
    \item 2.2 Conduct 1-week follow up trial.
  \end{itemize}
\end{enumerate}

\subsection{Tasks}

\subsubsection*{1. Building}
Tasks for building the chatbot are to implement:

\begin{enumerate}
  \item Interaction between a user and our system via the Facebook Messenger platform.
  \item Question-response statements for the chatbot.
  \item Training the chatbot on these responses.
  \item Reminder component for habit triggers.
  \item Visual rewards.
  \item Auditory rewards.
  \item Tactile vibration rewards by integrating with FitBit.
  \item On-boarding for new users to get started with the chatbot.
\end{enumerate}

\subsubsection*{2. Testing}

\begin{enumerate}
 \item Test the chatbot using a unit-test harness, to increase the quality of the chatbot.
 \item Write 4-week evaluation trial.
 \item Write 1-week follow up study.
 \item Conduct 4-week evaluation trial, users will use the chatbot daily to track a habit and perform a habit automaticity test before the study starts and after it has finished.
 \item Conduct 1-week follow up trial, users will stop using the chatbot, after 7 days will be contacted to perform a habit automaticity test and present how often they performed their habit in the 7 days.
 \item Analyse trial results.
\end{enumerate}

\pagebreak[4]\global\pdfpageattr\expandafter{\the\pdfpageattr/Rotate 90}
\begin{landscape}

\begin{figure}[ht]
    \begin{ganttchart}[hgrid, vgrid, y unit chart=.6cm, x unit=1cm]{1}{20}

      \gantttitle{2017, Academic Weeks}{20} \\
      \gantttitle{May}{5}
      \gantttitle{June}{4}
      \gantttitle{July}{5}
      \gantttitle{August}{4}
      \gantttitle{September}{2} \\
      \gantttitlelist{24,...,43}{1} \\

      \ganttgroup{Building}{1}{6} \\
      \ganttbar{Task 1.1}{1}{1} \ganttnewline
      \ganttlinkedbar{Task 1.2}{2}{2} \ganttnewline
      \ganttlinkedbar{Task 1.3}{2}{2} \ganttnewline
      \ganttlinkedbar{Task 1.4}{3}{3} \ganttnewline
      \ganttlinkedmilestone{
        Milestone 1.1. Chatbot.
      }{3} \ganttnewline


      \ganttbar{Task 1.5}{4}{4} \ganttnewline
      \ganttlinkedbar{Task 1.6}{4}{4} \ganttnewline
      \ganttlinkedbar{Task 1.7}{4}{6} \ganttnewline
      \ganttlinkedbar{Task 1.8}{6}{6} \ganttnewline

      \ganttlinkedmilestone{
        Milestone 1.2. Reward Delivery.
      }{6} \ganttnewline

      \ganttgroup{Testing}{7}{15} \\
      \ganttbar{Task 2.1}{7}{7} \ganttnewline
      \ganttlinkedbar{Task 2.2}{7}{7} \ganttnewline
      \ganttlinkedbar{Task 2.3}{8}{8} \ganttnewline
      \ganttlinkedbar{Task 2.4}{9}{12} \ganttnewline
      \ganttlinkedmilestone{
        Milestone 2.1. 30-Day Study.
      }{12} \ganttnewline

      \ganttbar{Task 2.5}{13}{13} \ganttnewline
      \ganttlinkedbar{Task 2.6}{14}{15} \ganttnewline
      \ganttlinkedmilestone{
        Milestone 2.2. 7-Day Study.
      }{15}

    \end{ganttchart}
\caption{Gantt Chart of estimated project timeline. Starts 01/05/17, ends 21/08/17}
\end{figure}
\end{landscape}
\pagebreak[4]\global\pdfpageattr\expandafter{\the\pdfpageattr/Rotate 0}

\subsection{Risk Analysis}
Problems that might occur in the project are summarised below with contingency plans and risk mitigation techniques.

\renewcommand{\arraystretch}{1.5} % Increase line height of the following tables

\begin{center}
\begin{tabular}{ |p{1.5cm}|p{1.9cm}|p{2.3cm}|p{3.3cm}|p{5cm}| }
  \hline \multicolumn{5}{|c|}{Main Project Risks} \\ \hline % Title
  \textbf{Severity} & \textbf{Likelihood} & \textbf{Risk} & \textbf{Prevention} & \textbf{Contingency} \\ \hline % Row
  5 & 1 & Chatbot could break during evaluation trial. & A unit-test harness will be constructed to increase the quality of the prototype. & Design recommendations can still be used from chatbot construction and if the chatbot doesn't completely break, the modalities that work can be tested against users. \\ \hline
  4 & 4 & Low sample size of users with wearable devices. & A popular wearable device has been chosen for integration to increase the number of users. & Additional wearable device integrations will be added, such as Jawbone, to increase pool of users. \\ \hline
  4 & 2 & Low sample size for evaluation trial. & Participants will be recruited ahead of time and university resources will be utilised. & Qualitative feedback will be gathered and focused upon, rather than quantitative to gleam a better understanding of design recommendations.\\ \hline
  3 & 2 & Under-estimate task length. & Contingency-weeks at the end of the project and overestimation of tasks allow for task slippage. & Modification of project objectives allows for a focus on HCI design recommendations for building chatbots that support habit formation. \\ \hline
\end{tabular}
\end{center}

