
\newpage
\section{Prototype Design}

To verify our hypotheses, a prototype was constructed to track habits and deliver rewards. The platform for the tool needed to be highly available for participants, interactive and time effective to build.
A system on a mobile device grants us access to a highly available, contextually aware and interactive platform~\cite{article_my_phone_is_part_of_my_soul, article_mhealth}. We will base the design on previous recommendations for building behaviour change systems.

\subsection{Existing Technology}

Current habit forming systems use apps to interact with users, but surveys of these systems~\cite{survey_on_apps_2,survey_on_current_apps_of_steel, article_mhealth} revealed most of them fall into a low behavioural theory adherence scale.
Research into how to build systems that are theory-based, suggest four main stages for designing health technology solutions. Conceptualisation, Formative Research and Pretesting,
Pilot trials and Evaluation trials~\cite{article_mhealth}. The first two stages use the Behaviour Change Wheel framework~\cite{article_behaviour_change_wheel}
(method for planning health behaviour interventions). This framework allows us to understand the behaviour, better define the characteristics and turn concept into prototype.
Pilot trials test the prototype before it is production-ready without the commitment of a full trial. The final stage evaluates the finalised prototype with a wider range of participants. Next, conceptualisation and design phases are discussed for design steps to build the prototype.

\subsection*{Conceptualisation and Formative Research}

This first stage explores different mediums to develop a prototype upon and different methods of user interaction. We conclude with a series of implementation options discussion about how the chatbot was chosen.\newline
\newline
When it comes to mobile phones, users have plenty of options for interaction. A popular choice that has revisited the market are \textit{chatbots}. Chatbots are applications that parse questions using Natural Language Processing (NLP) to provide a response. Bots act as a user interface to expose data and would use online services to parse the response, such as Amazon lex (\url{https://aws.amazon.com/lex/}). These programs have conversations with users to achieve a goal and are not new inventions. Since 1966~\cite{article_eliza}, Eliza by Joesphs Weizenbaum, used simple expression matching to return a certain response for user trials. In the present day, these applications (commonly referred to as bots or chatbots), are found integrated into many different apps on the majority of users mobile phones. For example, Facebook Messenger, a popular messaging application (\url{www.messenger.com}) encourages developers to create bots to interact with their users. These bots act as a real person with similar interaction flow, plus a few additional features, such as \textit{Quick Replies}~\cite{doc_fb_quick_replies} for revealing a list of options to a user. Quick Replies provide a way to present buttons to the user in response to a message. However, these bots would not reply like a real person, but rather would only reply if that question was pre-trained using machine learning algorithms. This technology requires the bot to be trained on a large set of data and the majority of use cases would have to be accounted for.
NLP would enable users to chat to the bot and get a friendly understandable reply. However, this interaction may build a dependence on the user-chatbot interaction and could lead to losing automaticity. NLP will not be used to limit the scope of this project and avoid these potential problems.
Instead of natural language processing, the location of the bot (inside an existing messaging app) ease of interaction and the additional features (Quick Replies) help easily communicate with users.


Another design option is building a \textit{webapp}. However, the functionality is limited, due to the lack of notification features in current webapps. TO allow notifications the webapp could be paired with another native app or use SMS notifications However, because these technologies are separated it is hard to get users to respond. Another design option is a native mobile app as has the ability to supply notifications, but for each platform a completely separate app would need to be built and users would need to download the app before it would be available to them. A single cross-platform app could be constructed to reduce development time and complexity, but still users would need to download the app to start using it. A web app has the advantage of being available to all users with a web browser (with users being able to save the site to home screen), but without notifications on all platforms (iOS only), it won't meet our requirements. Finally, a chatbot integrated into a popular messaging platform is easily available (if you have the messaging app already installed), simple (the user interface is already supplied), works on any platform the messaging app is available on and has notifications built in.

\renewcommand{\arraystretch}{1.5} % Increase line height of the following tables
\begin{figure}[H] % ht
\begin{center}
\begin{tabular}{ |p{3.8cm}|p{2.5cm}|p{4cm}|p{2.5cm}|p{2cm}| }
 \hline
 \textbf{} & \textbf{Mobile App} & \textbf{Cross-Platform App} & \textbf{Web App} & \textbf{Chatbot} \\ \hline
 Notifications & \cmark & \cmark & \xmark & \cmark \\ \hline
 Development Time & Long & Medium & Short & Short \\ \hline
 High Availability & \xmark & \xmark & \cmark & \cmark \\ \hline
 Simplicity & \xmark & \xmark & \cmark & \cmark \\
 \hline
\end{tabular}
\end{center}
    \caption{Comparing different prototype mediums.}
    \label{fig:prototype_table}

\end{figure}

The table (Figure~\ref{fig:prototype_table}) summarises the available development choices for a system that meets the recommendations (Section~\ref{recommendations}).

\subsection{Platform}

There are lots of options about what platform to build the chatbot into. For example, Slack (www.slack.com) bots provide additional functionality, such as habit track, into the popular workplace communication service.

\begin{figure}[H] % ht
    \centering
    \includegraphics[width=3.5in]{../resources/existing-bots/healthy-bot.png}
    \caption{Healthy Bot (\url{https://healthybot.io}): A Slack bot for forming new positive habits.}
    \label{fig:healthy_bot}
\end{figure}

However, if our main aim is to interact with lots of people easily, we need to target existing platforms that have high participant availability.

\begin{figure}[H] % ht
\begin{center}
\begin{tabular}{ |p{3.8cm}|p{4cm}|p{2.2cm}|p{1cm}|p{1.8cm}|p{1.3cm}| }
 \hline
 \textbf{} & \textbf{Facebook Messenger} & \textbf{WhatsApp} & \textbf{SMS} & \textbf{Telegram} & \textbf{Slack} \\ \hline
 High Availability & \cmark & \cmark & \cmark & \xmark & \xmark \\ \hline
 Interactive & \cmark & \cmark & \xmark & \cmark & \cmark \\ \hline
 Additional Features & \cmark & \xmark & \xmark & \cmark & \cmark \\
 \hline
\end{tabular}
\end{center}
    \caption{Comparing different chatbot platforms.}
    \label{fig:chatbot_platform_table}

\end{figure}

Facebook Messenger looks like the attractive option for user interaction with the ease of additional features, such as Quick Replies and with the benefit of:

\begin{itemize}
  \item 1,200 Million active users per month (as of April 2017)~\cite{fb_messenger_stats}
  \item Embedded into a service users already use
  \item Quick replies allows for easy interaction
\end{itemize}

The success of the chatbot will depend on how people differentiate between the bot and another contact and if people prefer the interaction.

\subsection*{Facebook Messenger}

\begin{figure}[H] % ht
    \centering
    \includegraphics[width=2.5in]{../resources/existing-bots/poncho.jpg}
    \caption{Poncho: An example of Facebook Messenger Weather Chatbot}
    \label{fig:poncho}
\end{figure}

Discussion about the additional features fb messenger has with discussion about previous chatbots.\newline
How people interact with facebook.


\begin{figure}[H]
  \centering
  \includegraphics[width=1.9in]{../resources/existing-bots/everist.png}
  \hspace{10px}
  \includegraphics[width=1.9in]{../resources/existing-bots/cleo.png}
  \hspace{10px}
  \includegraphics[width=1.9in]{../resources/existing-bots/joy-ai.png}
  \caption{Examples of chatbots performing different actions}
  \label{fig:chatbots_examples}
\end{figure}


\subsection*{Supporting Habit Formation}

Discussion about how we can support habit formation using a chatbot in reality.

Paragraph about Gamification elements:
- Gamification elements~\cite{article_free_to_play_making_money_from_games_you_give_away}
- Designing outstanding feedback loops~\cite{website_how_to_design_feedback_loops}

Justify the type of habits chosen: Because the evaluation trial is shorter, the habits must be easier - as the easier it is, the quicker it will form, if you can easily do it, you are more likely to do it


\subsubsection*{Bot}
The bot was designed to track participants habits and deliver the rewards. The bot was built into an existing social network participants were used to, therefore the social scaffolding around the interaction helped reinforce the response to notifications as participants are more likely to respond because participants perceive the bot as a real person because it is located next to other real people.

\subsection*{Delivering Rewards}

\begin{figure}[H]
  \centering
  \includegraphics[width=2.1in]{../resources/design/reward-audio-inline.png}
  \hspace{10px}
  \includegraphics[width=2.1in]{../resources/design/reward-visual-inline.png}
  \caption{In-line rewards.}
  \label{fig:rewards_inline}
\end{figure}

Discussion about how the rewards will be delivered to the users. In-line verses same screen. How interaction is handled. Concludes with exactly how rewards will be handled in design way, brief discussion about vibration and how the setup was handled, but would not work.

\begin{figure}[H]
  \centering
  \includegraphics[width=2.1in]{../resources/design/reward-visual-inline.png}
  \hspace{10px}
  \includegraphics[width=2.1in]{../resources/design/reward-visual-2.png}
  \caption{In-line visual rewards verses separate.}
  \label{fig:rewards_inline_v_website}
\end{figure}

The types of rewards are separated into three categories, visual, auditory and combined visual-auditory. Different types of these rewards will be experimented with,
and test if they provide user satisfaction. These rewards will be displayed to the user within the chatbot after they complete their habit.
For a playful prototype, visual rewards will be light-hearted gifs and auditory rewards were selected to match the gifs.

\begin{figure}[H]
  \centering
  \includegraphics[width=2.1in]{../resources/design/reward-visual-2.png}
  \hspace{10px}
  \includegraphics[width=2.1in]{../resources/design/reward-audio.png}
  \caption{Consistency between every reward, even across devices. Left, visual reward, right, auditory reward.}
  \label{fig:rewards_consistency}
\end{figure}

To ensure consistency with the reward types, a standard method was deigned to deliver the reward to users.
Instead of sending the rewards in-line with the messages, a \textit{webview} was used to display a website where users opened their reward.
This meant we could get out of the sandbox that the bot lives in, and use HTML5 elements to display the content in the same way, ensuring consistency across devices.
Although this was not without its limitations.

\subsection*{Design Limitations}

Auto-playing an audio track was not available when sending the audio in-line. But using a webview we could use the \textit{HTML5} audio element to enable auto-play.
But, for auto-play elements, the HTML5 standard needs a button press before it starts (TODO footnote, this is for a good reason, to stop people spamming you with audio when visiting sites, see more mozilla reference autoplay).
This required another button to create a JavaScript hook to auto-play.
However, during tests on low mobile data speeds, users found that they would have to press the button multiple times before the audio played.
This was because the audio would only play after it had loaded and created a lengthy delay, along with seemingly broken display.
To create a better user experience, the button disappeared when pressed, using CSS that would execute even if the audio hadn't loaded, and even on a poor connection.
Then JavaScript would execute after the page had fully loaded and play the audio if the button had already been pressed from the CSS, but if it hadn't then it would create a hook to play the audio after it had been pressed.
This ensured a seemly experience when using rewards for all levels of connection.

\subsection{User Flow}

Full user interaction from setup to completing a reward---Logo design by the Noun Project by Yu luck (\url{https://thenounproject.com/term/custom/402041/}).

\begin{figure}[H]
  \centering
  \includegraphics[width=1.5in]{resources/design/process/1.jpg}
  \hspace{10px}
  \includegraphics[width=1.5in]{resources/design/process/2.jpg}
  \hspace{10px}
  \includegraphics[width=1.5in]{resources/design/process/3.jpg}
  \hspace{10px}
  \includegraphics[width=1.5in]{resources/design/process/4.jpg}
  \newline
  \newline
  \includegraphics[width=1.5in]{resources/design/process/5.jpg}
  \hspace{10px}
  \includegraphics[width=1.5in]{resources/design/process/6.jpg}
  \hspace{10px}
  \includegraphics[width=1.5in]{resources/design/process/7.jpg}
  \hspace{10px}
  \includegraphics[width=1.5in]{resources/design/process/8.jpg}
  \newline
  \newline
  \includegraphics[width=1.5in]{resources/design/process/9.jpg}
  \hspace{10px}
  \includegraphics[width=1.5in]{resources/design/process/10.jpg}
  \hspace{10px}
  \includegraphics[width=1.5in]{resources/design/process/11.jpg}
  \hspace{10px}
  \includegraphics[width=1.5in]{resources/design/process/12.jpg}
  \caption{Setup flow for Harry's Habits.}
  \label{fig:setup_flow_screenshots}
\end{figure}


\begin{figure}[H]
  \centering
  \includegraphics[width=1.5in]{resources/design/process/13.jpg}
  \hspace{10px}
  \includegraphics[width=1.5in]{resources/design/process/14.jpg}
  \hspace{10px}
  \includegraphics[width=1.5in]{resources/design/process/15.jpg}
  \caption{Continued: Setup flow for Harry's Habits.}
  \label{fig:setup_flow_screenshots_2}
\end{figure}

Talk about designing the flow, differences between free text and quick replies.

\begin{figure}[H]
  \centering
  \includegraphics[width=2.1in]{resources/figures/reminder.png}
  \hspace{10px}
  \includegraphics[width=2.1in]{resources/figures/reward-visual.png}
  \caption{Harry's Habits delivering a reminder then a visual reward.}
  \label{fig:setup_media_1}
\end{figure}

\begin{itemize}
  \item \textbf{Setup}
  \begin{itemize}
    \item Press button that opens the bot in the Facebook Messenger app.
    \item Answer a series of setup questions to the bot via the Facebook Messenger app.
    \item User chooses an existing habit they would like to develop from a list.
    \item User supplies an existing routine to integrate their new habit into.
    \item User chooses a time the existing routine normally happens.
    \item User finishes the setup and closes the Facebook Messenger app.
  \end{itemize}
  \item \textbf{Trigger}
  \begin{itemize}
    \item At the time of the existing routine the user performs their chosen habit.
    \item The user receives a notification after the routine, asking if they managed to complete their habit or if they need more time.
    \item If they need more time, the notification will \textit{snooze} for about an hour and be sent again.
    \item If users regularly snooze they will be asked if the time of their existing routine has changed.
    \item If users say they have completed their habit, they will be sent a message Quick Reply that leads to a reward.
  \end{itemize}
  \item \textbf{Reward}
  \begin{itemize}
    \item Users will press the reward button that will take them to a website that contains a link to open a reward. This allows the experience to be consistent for each reward modality.
    \item User will receive a reward from one of the following modalities:
    \begin{itemize}
      \item Visual: video with no sound
      \item Audio: soundtrack
      \item Visual-Auditory: video with sound
    \end{itemize}
  \end{itemize}
\end{itemize}

\newpage
