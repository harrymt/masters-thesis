
\newpage
\section{Conclusion}

The survey on the relevant literature demonstrates two sets of design requirements for building systems that support habit formation.
In particular building habit automaticity is key to building lasting behaviour change and multi-cue routines better support behaviour change than single cues.
Rewards from different modalities are key to building habit automaticity with interaction from different modalities increasing user interaction.
These combine to form new requirements that focus on delivering rewards from different modalities.\newline
\newline
A chatbot is designed and constructed from these requirements to track habits and deliver rewards to users from different modalities.
A 3-week evaluation trial tested the effectiveness of the chatbot implementation, the design requirements and if bots are a good method for collecting research data.
Finally a 1-week follow up trial tested user's habit automaticity.\newline
\newline
Evaluation from real users revealed important positive and negative aspects of the requirements.
Following up after the study determined if the requirements were effective for building habit automaticity and testing the validity of our hypothesis.\newline
The project found ....\newline
\newline
Therefore, the project presents a novel method of interacting with users to track habits and a system evaluation provides value
on how to build a chatbot to deliver rewards from different modalities to support habit formation.
Finally, the project opens up new research avenues for investigating the use of chatbots as vehicles for promoting behaviour change.
