
\section{Limitations and Future Work} \label{limitations_and_future_work}
There were several limitations to these findings. Firstly, the evaluation of the first hypotheses relied on participants self-reporting. Participants could have simply lied to remove the alert and get the reward. This is particularly true with the snooze function, as some participants found it annoying and stopped using the chatbot over time. Participants may of been getting rid of the reminder rather than completing the habit, therefore the measurement could have been how participants reacted to notifications instead of if they competed their habit. Therefore, it is difficult to draw any valid conclusion on actual habit performance. Future work into how quickly participants responded to the alerts and if the device delivery (browser, iOS or Android) would better understand how participants interacted with the alerts. Secondly, only 7 participants responded to the follow up interviews for the evaluation of the second hypotheses. In addition, the study relied on participants recall, which could be inaccurate. Therefore, a larger sample size for the SRBAI questionnaire is needed to validate this hypotheses and the findings. Thirdly, additions to the chatbot, such as using natural language processing, could improve the functionality of the chatbot, proving a more conversational user interface and perhaps could increase participant interaction and perhaps not create a dependence between our prototype and participants habit performance during the 3-week period as this was disadvantage for habit formation. Fourthly, the small number of participants and the small sample of rewards used in the study make it unclear how these findings would generalise to other types of rewards with the same modality. Finally, these findings only apply to the intrinsic positive reinforcement rewards used in the study, as the content and method of delivery is another variable that effects these results. Additional studies using different types of bot-delivered rewards from different modalities would validate these findings.

\newpage

\section{Conclusion}
We have surveyed three areas of literature: habit formation rewards, different modes of feedback and chatbot interaction. We found that habit performance and habit automaticity are keys to building lasting behaviour change. This thesis builds these three areas into a set of design recommendations for building reward-based chatbots that support habit formation.
A prototype is designed and constructed from these recommendations, that aims to track habits and deliver rewards from three modalities: visual, auditory and visual-auditory combined.
We evaluated our prototype during a 4-week study against two hypotheses, comparing these against a control group and each reward type: i) How is habit performance effected. ii) How is habit automaticity effected. 36 participants completed the study interacting with the bot for 3 weeks to try and form a new positive habit, then for a further 1-week interaction with the prototype was suspended. Validated habit automaticity questionnaires and 7 participant interviews were performed for further validation. This allowed us to evaluate how habit performance and habit automaticity was effected without technology. The results show how rewards delivered by Harry's Habits affected habit performance and habit automaticity.
More specifically,
the results found that the bot-delivered rewards improved habit performance. Participants were more likely to complete their habit if given the reward. Habit performance was also effected by different modalities, although not in the way our hypotheses assumed, as singular modalities had higher habit performance than visual-auditory rewards. Finally, the limitations of the study do not show any clear statistical significance whether the rewards or the combined modalities effected habit automaticity. More conclusive evidence is needed to show that rewards from combined modalities effect habit automaticity. Using these results to compare different types of visual, auditory and visual-auditory rewards with behaviour change technology may open up new research avenues for investigating the use of bots as tools to help form new habits.

