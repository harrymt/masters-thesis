
\newpage

\section{Limitations and Future Work} \label{limitations_and_future_work}
There were several limitations to these findings. First, the small number of participants and the small sample of rewards used in the study make it unclear how these findings would generalise to other types of rewards with the same modality. Second, this only applies to intrinsic positive reinforcement rewards, further research into how different types of rewards from different modalities is needed. Third, we outline the dependence between our prototype and participants habit performance during the 3-week period and how this has disadvantages for habit formation. Fourth, TODO: Vibration limitations. Fifth, TODO: using wit.ai for NLP. Finally, the content and method of delivery is another variable that effects these results, additional studies into bot-delivered rewards would validate these findings.


\section{Conclusion}
We have surveyed three areas of literature: habit formation rewards, different modes of feedback and chatbot interaction. We found that habit performance and habit automaticity are keys to building lasting behaviour change. This thesis builds these three areas into a set of design recommendations for building reward-based chatbots that support habit formation.
A prototype is designed and constructed from these recommendations, that aims to track habits and deliver rewards from three modalities: visual, auditory and visual-auditory combined.
We evaluated our prototype during a 4-week study against two hypotheses, comparing these against a control group and each reward type: i) How is habit performance effected. ii) How is habit automaticity effected. 36 participants completed the study interacting with the bot for 3 weeks to try and form a new positive habit, then for a further 1-week interaction with the prototype was suspended. Validated habit automaticity questionnaires and 7 participant interviews were performed for further validation. This allowed us to evaluate how habit performance and habit automaticity was effected without technology. The results show how rewards delivered by Harry's Habits effected habit performance and habit automaticity. More specifically, the results found that the bot-delivered rewards improved habit performance. Participants were more likely to complete their habit if given the reward. Habit performance was also effected by different modalities, although not in the way our hypotheses assumed, as singular modalities had higher habit performance than visual-auditory rewards. Finally, the limitations of the study do not show any clear statistical significance whether the rewards or the combined modalities effected habit automaticity. More conclusive evidence is needed to show that rewards from combined modalities effect habit automaticity. We encourage the use of these findings with further comparisons against different types of rewards for behaviour change technology. Finally, we hope this opens up new research avenues for investigating the use of bots as vehicles for promoting behaviour change and forming new habits.
