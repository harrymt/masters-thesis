
\newpage
\section{Conclusion}

The survey on the relevant literature demonstrates two sets of design requirements for building systems that support habit formation. In particular building habit automaticity is key to building lasting behaviour change and multi-cue routines better support behaviour change than single cues. Rewards from different modalities are key to building habit automaticity with interaction from different modalities increasing user interaction. These combine to form new requirements that focus on delivering rewards from different modalities.\newline
\newline
This will be a successful project because the requirements are based on theory. This gives us a chance of success for the implementation, as long as the construction adheres to those requirements.\newline
\newline
A chatbot designed from these requirements will deliver rewards to users from different modalities. A 4-week evaluation trial will test the effectiveness of the chatbot implementation and the design requirements. Finally a 1-week follow up trial will test user's habit automaticity.\newline
\newline
Evaluation from real users testing the implementation reveals important positive and negative aspects of the requirements. Following up after the study plays an important part in determining if the requirements were effective for building habit automaticity and testing the validity of our hypothesis. If user's habit automaticity does not increase, the project still presents a novel method of interacting with users to track habits and a system evaluation provides value on how to build a chatbot to deliver rewards from different modalities to support habit formation. Finally, the project opens up new research avenues for investigating the use of chatbots as vehicles for promoting behaviour change.
