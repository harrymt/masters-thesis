
\newpage
\section{Conclusion}

The survey on the relevant literature demonstrates two sets of design requirements for building systems that support habit formation. In particular; Building habit automaticity is key to building lasting behaviour change; Multi-Cue routines better support behaviour change than single cues; How rewards are key to building habit automaticity; Finally how multi-modal interaction increases user interaction. These combine to form new requirements that focus on delivering multi-modal rewards.\newline
\newline
This will be a successful project because the requirements are based on theory. This gives us a strong standing for the success of the implementation, as long as the construction adheres to those requirements.\newline
\newline
A chatbot designed from these requirements delivers rewards to users. A 30 day user study tests the effectiveness of the chatbot implementation and the design requirements. Finally a 7 day follow up study tests users habit automaticity.\newline
\newline
Evaluation from real users testing the implementation reveals important positive and negatives aspects of the requirements. Following up after the study plays an important part in determining if the requirements were effective for building habit automaticity and testing our hypothesis. accepted. If users do not increasing habit-automaticity, the project still presents a novel method of interacting with users to track habits and a system evaluation provides value on how to build a chatbot to deliver multi-modal rewards to support habit formation.
