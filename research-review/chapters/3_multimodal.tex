
\newpage

\section{Multi-Modal}
Research into how different modalities affect peoples behaviour shows us multiple contextual cues better support behaviour change, compared with a single mode \cite{article_understanding_use_contextual_cues_design_impl}. This can be combined with crossmodal interaction. The process of signals we receive through a single sense affecting how we process information perceived through a different sense \cite{article_natural_cross_modal_mappings}. This can be used by testing rewards delivered from different modalities, to see if the crossmodal interaction differs when the modality changes.\newline
\newline
Interacting with a user with a different modality might affect their behaviour in a different way. Therefore, each type of modality will be experimented with to test if the users behaviour is affected. The next section discusses methods of interaction in practice.

\subsection{Why Multi-Modal Interaction}
Research \cite{article_user_centred_multimodal_reminders} into unimodal reminders (triggers) systems compared with multi-modal, suggests a need for highly flexible and contextualised multi-modal and multi-device reminder systems. Although this study focused on the elderly, and the need to multiple modalities was important because some peoples sensory modalities decline with age, this principle still holds true for general case reminder systems. The study presents design guidelines for reminder systems. These are mainly focused on users needing a choice of modalities for interacting with users, as users want a highly configurable system. These aspects will be implemented into our project and adapted for delivering rewards.

\subsection{Modality Types}
Research into designing multi-modal reminders for the home, states that `good reminder systems should be multi-modal, because they provide alternative ways to interact with a user. Using multi-modal interaction could increase both accessibility of information being presented and the likelihood that the delivery method will be pleasant or acceptable to the user without becoming disruptive or annoying' \cite{article_designing_multimodal_reminders_for_home}.

\subsubsection*{Visual}
    - What is it
    - Why are we choosing it
    - Examples
    \subsubsection*{- Why is it good?}
      - @TODO research
    \subsubsection*{- Reward}
      - Cross check with my requirements

  \subsubsection*{Auditory}
      - What is it
      - Why are we choosing it
      - Examples
      \subsubsection*{- Why is it good?}
        - What does it give to us
        - @TODO research
      \subsubsection*{- Reward}
        - Cross check with my requirements
  \subsubsection*{Tactile}
      - What is it
      - Why are we chosing it
      - Examples
      - Wearables, Fitbit
    \subsubsection*{Why is it good?}
        - @TODO research
    \subsubsection*{Reward}
        - Cross check with requirements

\subsection{Delivering Rewards}
  - Table of multimodal reward implementation strategies
  - Table of requirements matched with modalities
  - If you could implement this, you could increase users automaticity for habits

