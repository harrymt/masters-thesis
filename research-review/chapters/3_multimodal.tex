
\newpage

\section{Multi-Modal}
Research shows multiple contextual cues better support behaviour change, compared with a single mode \cite{article_understanding_use_contextual_cues_design_impl}. If we combine this with crossmodal interaction - The process of signals we receive through a single sense affecting how we process information perceived through a different sense \cite{article_natural_cross_modal_mappings}, we can interact with users on different modalities. This can be used by testing rewards delivered from different modalities, to see if the crossmodal interaction differs when the modality changes.\newline
\newline
Interacting with a user with a different modality might affect their behaviour in a different way. Therefore, each type of modality will be experimented with to test if the users behaviour is affected. The next section discusses methods of interaction in practice.

\subsection{Why Multi-Modal Interaction}
@TODO add references!

Research \cite{article_user_centred_multimodal_reminders} into unimodal reminders (triggers) systems compared with multi-modal, suggests a need for highly flexible and contextualised multi-modal and multi-device reminder systems. Although this study focused on the elderly, and the need to multiple modalities was important because some peoples sensory modalities decline with age, this principle still holds true for general case reminder systems. The study presents design guidelines for reminder systems. These are mainly focused on users needing a choice of modalities for interacting with users, as users want a highly configurable system. These aspects will be implemented into our project and adapted for delivering rewards.

\subsection{Modality Types}
Research into designing multi-modal reminders for the home, states that `good reminder systems should be multi-modal, because they provide alternative ways to interact with a user. Using multi-modal interaction could increase both accessibility of information being presented and the likelihood that the delivery method will be pleasant or acceptable to the user without becoming disruptive or annoying' \cite{article_designing_multimodal_reminders_for_home}. @TODO add so what???

\subsubsection*{Visual}
One study looked at improving habit consistency for how often patients took medication, by using a visual display device that gave constant feedback \cite{article_realtime_feedback_improving_medication_taking}. They found that this feedback improved consistency of the habit and increasing rating of self-efficacy. But when the device was removed, their performance dropped (2 months follow up study), because users integrated the feedback display with their routines. This habit-forming system used visual feedback to encourage consistency, however, this system shouldn't integrate visual cues into the system, otherwise users will become dependent on the technology. Users should instead build these cues outside of the system to build performance longevity after removing the system.

\subsubsection*{Auditory}
Another key paper, discussed their need for multimodal applications when designing for the elderly \cite{article_movipill_improving_medication_elders}, combing different sounds with high visual contrast to suit their needs, given deteriorating senses due to age. The study showed that multimodal interaction gives a means of communication to people with varying levels of sense ability. But studies have also shown people need a choice of mode when designing for multimodal interaction \cite{article_user_centred_multimodal_reminders}, and thus the design requirements produced from this study can be applied to general multimodal applications, such as this project. The design guidelines discuss the need for interaction consistency, such as using similar audio interaction. @TODO therefore the mapping between the visual, auditory and tactile will be mapped to a similar pattern.


\subsubsection*{Tactile Vibration}
The majority of electronic activity monitors have behaviour change techniques and these monitors present a medium which behaviour change interventions could occur \cite{article_wearable_good}. This provides us with a final modality to explore reward delivery techniques - implemented with a wearable device using vibration. One of the activity monitors researched, the Fibit, will be the primary platform for integrating a Tactile mode for rewards, due to @TODO better phrase this: researcher availability.

\subsection{Reward Types}
Rewards will be delivered from each modality, with the content based on the habit-forming requirements listed above. Visual rewards will present the user with a photo that gives users satisfaction of completing the habit, auditory rewards will provide a similar result but via the auditory mode, finally the vibration patterns shall cater towards the tactile mode. The next section discusses design recommendations for delivery methods for these rewards.

