
\section{Introduction}

\subsection*{Background}
People have goals they want to achieve that require repetitive actions, such as regular exercise or losing weight. Habits can be used to perform these actions with almost no conscious thought in a automatic-like way. Forming a positive habit can increase the chance people achieve these type of goals by changing their behaviour \cite{article_promoting_habit_formation}. Evidence suggests there are three elements of habit formation, repetition, contextual cues and positive reinforcement \cite{article_beyond_self_tracking_designing_apps}. Associating the cue with performance and grounding the process with a reward encourages regular repetition, leading to automatic behaviour \cite{article_experiences_of_habit_formation}. Building a new habit requires a contextual cue, to trigger the start of the habit (action), and a reward for positive reinforcement \cite{article_beyond_self_tracking_designing_apps, article_how_habits_formed_modelling_habit_formation}. For example, a reminder on your phone (trigger) reminds you to stretch (action), relaxing you and removing back pain (reward). Studies have shown that the process of creating a new habit takes on average up to 66 days of repetitive use \cite{article_how_habits_formed_modelling_habit_formation}. However, anyone who has ever made a new years resolution knows the difficulties in changing behaviour. Too many people try to create new positive habits only to drop them a few days into their new routine.

\subsection*{Motivation}

\subsubsection*{Habit Formation}
Technology can solve this problem, by coaching us through a new routine action until it becomes a habit. Mobile technology can issue reminders and build motivation for repetitive tasks. They provide us with an interactive platform that can help support habit formation. Plenty of existing habit formation systems use apps that guide users through a series of experiences to form a new habit. However, literature shows these apps are unsuccessful at forming habits because they are not grounded in habit formation theory \cite{article_beyond_self_tracking_designing_apps, article_apps_of_steel}. The apps create a dependence on the technology and do not build the automatic reaction to a trigger (habit automaticity) \cite{article_dont_kick_habit} e.g. when people stop using habit forming mobile apps, they also stop performing the habit.\newline
\newline
Studies have shown that routine-based remembering strategies are good for building habit automaticity. Stawarz et al. \cite{article_dont_forget_your_pill} produced a set of design requirements for building mobile apps grounded in habit formation theory that aim to build habit automaticity. Other literature \cite{article_taxonomy_motivational_affordances_meaningful} shows how intrinsic rewards build habit automaticity, producing a set of real-world implementation requirements for building habit formation systems. These requirements will combine to form a new set focused on rewards and this project will use these requirements to implement a tool that builds habit automaticity.

\subsubsection*{Chatbots}
Interaction with current habit formation systems is often via a mobile app. This creates a notable difference in the person when the system is removed, as people personalise their phone and it becomes a part of them \cite{article_my_phone_is_part_of_my_soul}. This is also the case with many mobile feedback systems that aid with behaviour change, when the system is removed any improved performance does not persist \cite{article_dont_kick_habit, article_realtime_feedback_improving_medication_taking}.\newline
\newline
Chatbots are a method of communicating with a computer system via a conversation using natural language. They provide a better mobile phone integration for users, as they hook into messaging services users are familiar with. When removing the system, instead of removing an app, users stop messaging a person (the chatbot). This project will build a chatbot to deliver reminders and rewards to users, acting as a novel tool to facilitate interaction with users.

\subsubsection*{Reward Modalities}
Habit formation systems use reminders and rewards. Studies have shown that good reminder systems should be multi-modal \cite{article_designing_multimodal_reminders_for_home}, that is providing alternative ways to interact with the user, either visual, auditory or tactile vibration. This increases the likelihood that the delivery method will be pleasant and satisfactory to the user. This project will use this idea for rewards, incorporating this technique into the chatbot by delivering rewards to users across multiple modalities. However, reminders will still be issued on a single mode (visual) to limit the scope of this project to test how rewards from different modalities effect peoples habit strength.
@TODO cite more modalities

\subsection*{Aims}
This project aims to evaluate how users habit strength is effected by rewards using a chatbot that delivers reminders and rewards to users. The reminders will be on a single mode and the rewards will be from different modalities.

\subsection*{Objectives}
The chatbot will provide habit tracking with reminder messages as triggers, and rewards as positive reinforcement in three modalities, visual, auditory and tactile vibration.\newline
\newline
The rewards will provide the user with the satisfaction of completing the habit action and encourage them to build user habit automaticity. The visual reward will be a photo, the auditory reward will be an audio clip and the tactile reward will integrate with a wearable to provide tactile feedback from vibrations. Each reward mode content will be comparable with another, using crossmodal mapping across all rewards modalities. Different types of mappings will be explored during the implementation of rewards. For example, a visual picture of a bird could be mapped with the sound (auditory) of a bird, and a tactile vibration pattern similar to the wavelengths of the bird sound.

\subsection*{Methodology}
The literature review enabled the construction of theory-based requirements that are focused on rewards for habit formation. Next a tool can be built that uses these requirements, to ensure that the system is based on theory. A chatbot will be constructed as a tool to track habits, deliver reminders as notifications and deliver rewards from three modalities, visual, auditory and tactile vibration. To utilise participant availability, the chatbot will be built using a popular messaging platform, Facebook Messenger.\newline
\newline
A 4-week evaluation trail will test the success of the chatbot by evaluating the tool and the effectiveness of each modality on users habit strength using a validated questionnaire. Chatbot interaction will be removed during a 1-week follow up trail to test if users continue with the habit. Participants will split into four groups, all groups will receive reminders, three groups will receive rewards each from a different modality, and one group (control group) will not recieve any rewards.

\subsection*{Deliverables}
To summarise, these are the following key project deliverables.

\begin{itemize}
  \item Chatbot that tracks habits by delivering reminders and rewards from three modalities, visual, auditory and tactile vibration.
  \item Analysis about how rewards from different modalities effect habit strength.
  \item Design recommendations for building chatbots that deliver rewards in different modalities.
\end{itemize}

\subsection*{Added value}
Evaluation from real users testing the implementation reveals important positive and negatives aspects of the requirements. Following up after the study plays an important part in determining if the requirements were effective for building habit automaticity and testing the validity of our hypothesis. If users habit automaticity does not increase, the project still presents a novel method of interacting with users to track habits and a system evaluation provides value on how to build a chatbot to deliver rewards from different modalities to support habit formation. Finally, the project opens up new research avenues for investigating the use of chatbots as vehicles for promoting behaviour change.\newline
\newline
This research review looks at the literature around habit formation and rewards from different modalities, summarising with design guidelines and a project plan.

\newpage

