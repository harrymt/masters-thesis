

\section{Executive Summary}

(short version of introduction)


\section{Introduction}

\subsection*{What is a Habit}

Habits are actions that are triggered automatically in response to a contextual cue. Associating the cue with performance and grounding the process with a reward encourages regular repetition, leading to automatic behaviour. For example, automatically washing your hands (action) after using the toilet (contextual cue) accomplish clean hands (reward). Once initiation of the action is transferred to external cues, dependence on concious attention or motivational processes is reduced. [cite : Experiences of habit formation: a qualitative study. Lally P, Wardle J, Gardner B. Psychol Health Med. 2011 Aug; 16(4):484-9. https://www.ncbi.nlm.nih.gov/pmc/articles/PMC3505409/#b10]

\subsection*{Positive Habits}

Literature shows us the health principles with forming positive habits.[ cite 2. Lally P, Gardner B. Promoting habit formation. Health Psychol Rev. In press: DOI: 10.1080/17437199.2011.603640. https://www.ncbi.nlm.nih.gov/pmc/articles/PMC3505409/#b12] However, the process of creating a habit takes on average 66 days [cite 66 day person ] of repetitive use, and requires a methodical process, starting with a trigger and ending with a reward.

\subsection*{What are Habit Rewards}

:\newline
:

\subsection*{What has technology got to do with this?}

Technology can change peoples behaviour and encourage them to form a habit by guiding users through a series of experiences. Trigger, Action and Reward. Theories have explored different methods of providing people with rewards, however, literature shows that these implementations are never as affective as the theory states.

\subsection*{What are some of the techniques technology uses?}

:\newline
:

\subsubsection*{Gamification}
Succeed better w gamification, look at dependancy w other things, clash royale
Technology can support habit formation. Gamification is a popular technique to accomplish this. [cite : Gamification Definition, http://gamification-research.org/wp-content/uploads/2011/04/01-Deterding-Sicart-Nacke-OHara-Dixon.pdf]

\subsubsection*{More modalities}
TODO find reason why this should be here


\subsection*{Current state of technology usage}

However, a recent review of current technology suggest that the implementations differ from the theory as all fail to support development of automaticity. [cite : Kathys, Beyond self-tracking and reminders, designing smartphone apps that support habit formation.]
- Dependency on systems
- Event based cues are better!

\subsection*{Why Build a System / Contributions / Added value}

- And focus on reward delivery [ cite kathys study, wasnt affective ]
Therefore, this project aims to use technology to develop a habit forming system that provides the ability to perform a positive habit as if it was as automatic as washing your hands after using the toilet. // This project focuses on habit reward systems and how theoretical methods of sustaining habits are put into practice. This project aims to implement different types of reward deliveries through different modalities and also provides a set of design recommendation for building habit forming systems.

\subsection*{Summary}

:\newline
:

\subsubsection*{Motivation}
Maintaining habit behaviour when the system is removed is difficult! We try to counter this and we can feasibly find a way for people to adopt a habit without relying on a system.


\subsubsection*{Project Type}
In this paper, we propose and evaluate three configurations for habit adoption.
The proposed project is split 40/60 between researching different configurations and implementing three configurations.


\subsubsection*{Aims and Objectives}
The aim is to achieve habit adoption when a configuration is removed.


\subsubsection*{Methodology}
Prior research into different habit adoption systems will refine the design requirements for three system configurations. System A, B and C.\newline
Will use design requirements from \cite{article_dont_forget_your_pill}.
These systems will be constructed using a staggered approach to optimise the use of time.
After testing of each system configuration, each individual system will be evaluated and the results will be compared to produce a set of implications for the design of a system.


\subsubsection*{Deliverables}
Three evaluated systems to adopt habits. Design requirements from the results of testing these three systems.


\subsubsection*{Added Value}
Results of studies and design requirements.


\subsection*{Old}
Old: Systems to assist with habit adoption are intended to start new behaviours and try to maintain the same behaviour even when users disengage with the system. Existing systems aimed at improving habit adoption use single modality constructs to enforce habit behaviour, like visual cues or vibrating alarm reminders. However, prior research has established a dependency between on-going habit adoption system use and lasting change. In this paper, we propose and evaluate three systems that use different modality configurations for habit adoption in conjunction with gamification. The aim is for users to keep engaged with the habit adoption when the system is removed. A 30 day user study will be conducted to evaluate the use of each system and test how users compliance with conducting the habit improves when the system is in place and then again when the system is removed. We conclude with an implementation and evaluation of the three proposed systems for habit adoption and present a set of implications for the design of a system in this domain.

