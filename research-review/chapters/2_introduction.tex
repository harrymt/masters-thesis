
\section{Introduction}

\subsection*{Background}
Habits are actions performed with almost no conscious thought in a automatic-like way. People want to form positive habits for lots of reasons (Lally, Promoting habit formation \cite{article_promoting_habit_formation}), for example health benefits, such as remembering to exercise everyday or medicinal benefits, such as remembering to take medication. The 3 elements of habit formation are, repetition, contextual cues and positive reinforcement (Kathys, 2.3.2). Associating the cue with performance and grounding the process with a reward encourages regular repetition, leading to automatic behaviour (Lally, Experiences of habit formation \cite{article_experiences_of_habit_formation}). Building a new habit requires three steps. A contextual cue, that triggers the start of the habit (action), then finishes with a reward with positive reinforcement (Kathys disso). For example, when you eat breakfast (trigger), you might write in a journal (action), and then reflect on last week (reward). Studies have shown that the process of creating a new habit takes on average up to 66 days of repetitive use (Lally). This is a long time for people to remember, perform and sustain a new habit without any help.

\subsection*{Motivation - Habit Formation}
Technology can aid this problem, by reminding and building motivation. Mobile phones provide us with an interactive platform that can help with habit formation. Existing habit formation apps guide users through a series of experiences to form a new habit. Although, literature shows these apps are bad at forming habits because they are not grounded in habit formation theory (Kathys: Beyond self tracking reminders). They get people dependent on the technology and do not build the automatic reaction to a trigger (habit automaticity). For example, when people stop using habit forming mobile apps, they also stop performing the habit.\newline
\newline
Studies have shown that routine-based remembering strategies are good for forming ?automaticity? (Kathy disso, table 7.1, also \cite{article_dont_forget_your_pill}) has produced a set of design requirements for building mobile apps grounded in habit formation theory. These requirements aim to facilitate the formation of reliable routine-based remembering strategies.

\subsection*{Motivation - Chatbot}
Interaction is often via a mobile app. This creates a notable difference in the person when removing the app from the phone, as people personalise their phone and it becomes a part of them (My phone is my soul). Removing the habit formation app, has shown to stop people performing the habit. Chatbots are a method of communicating with a computer system via a conversation using natural language. They provide a better mobile phone integration for users, as they hook into messaging services users are familiar with. When removing the system, instead of removing an app, users stop messaging a person (the chatbot). This could be the answer for solving user habit automaticity.

\subsection*{Motivation - Multi-Modal}
[Reward theory is mainly based on a single modality, ]
Studies (Designing multimodal reminders for the home, pg2, par1) have shown that good reminder systems should also interact with the user across different modalities. This would allow delivery of reminders and rewards across a modality to suit a user.

\subsection*{Motivation}
Therefore, a successful habit formation system should build habit automaticity via multimodal interaction over 66 days. After removing the system users should continue to perform the habit.

\subsection*{Aims}
This project aims to build a successful habit formation chatbot that uses multi-modal interaction.

\subsection*{Objectives}
The chatbot will provide habit tracking with reminder messages as triggers, and rewards in three modalities, visual, auditory and haptic.\newline
\newline
The rewards will provide the user with the satisfaction of completing the habit action and encourage them to build user habit automaticity. The visual reward will be a photo, the auditory reward will be a audio clip and the haptic reward will integrate with a wearable to provide tactile haptic feedback.

\subsection*{Methodology}
To net the largest amount of users to test, the chatbot will be implemented using a popular messaging platform, Facebook Messenger.\newline
\newline
A 66 day user study, and a 1 week follow up study, will test the success of the chatbot by evaluating effectiveness of each modality on habit automaticity. Chatbot interaction will be removed during the follow up study to test if users continue with the habit. Three groups, and a control group, will each receive reminders and rewards from a different modality.\newline
\newline
The user study will gather the following:
\begin{itemize}
  \item quantitative analysis of chatbot interaction
  \item qualitative survey of habit interaction at the end of the study, and end of follow up study
  \item automaticity test at beginning and end of study
\end{itemize}

\subsection*{Deliverables}
This project will produce the following:\newline

\begin{itemize}
  \item A chatbot that tracks habits and provides the above rewards
  \item Analysis of the effectiveness of different modalities on habit automaticity
  \item Design recommendations for building a habit formation chatbot
\end{itemize}

\subsection*{Added value}
The design recommendations will assist the HCI community for building habit forming chatbot systems. The user study will show how effective chatbots are and show how different types of reward deliveries through different modalities effect user habit automaticity.\newline
\newline
This research review looks at the literature around multi-modal rewards and habit formation, summarising with design guidelines and a project plan, to test if multi-modal rewards provided by a chatbot support habit formation.

\newpage

