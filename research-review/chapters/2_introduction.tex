
\section{Introduction}

\subsection*{Background}
People have goals they want to achieve that require repetitive actions, such as regular exercise or losing weight. Habits can be used to perform these actions with almost no conscious thought in a automatic-like way. Forming a positive habit can increase the chance people achieve these type of goals by changing their behaviour \cite{article_promoting_habit_formation}. There are three elements of habit formation, repetition, contextual cues and positive reinforcement \cite{article_beyond_self_tracking_designing_apps}. Associating the cue with performance and grounding the process with a reward encourages regular repetition, leading to automatic behaviour \cite{article_experiences_of_habit_formation}. Building a new habit requires a contextual cue, to trigger the start of the habit (action), and a reward for positive reinforcement \cite{article_beyond_self_tracking_designing_apps, article_how_habits_formed_modelling_habit_formation}. For example, a reminder on your phone (trigger) reminds you to stretch (action), relaxing you and removing back pain (reward). Studies have shown that the process of creating a new habit takes on average up to 66 days of repetitive use \cite{article_how_habits_formed_modelling_habit_formation}. However, anyone who has ever made a new years resolution knows the difficulties in changing our behaviour. Too many people try to create new positive habits only to drop them a few days into their new routine.

\subsection*{Motivation}

\subsubsection*{Habit Formation}
Technology can solve this problem, by coaching us through a new routine action until it becomes a habit. Mobile technology can issue reminders and build motivation for repetitive tasks. They provide us with an interactive platform that can help support habit formation. Plenty of existing habit formation systems use apps that guide users through a series of experiences to form a new habit. However, literature shows these apps are unsuccessful at forming habits because they are not grounded in habit formation theory \cite{article_beyond_self_tracking_designing_apps, article_apps_of_steel}. The apps create a dependence on the technology and do not build the automatic reaction to a trigger (habit automaticity) \cite{article_dont_kick_habit} e.g. when people stop using habit forming mobile apps, they also stop performing the habit.\newline
\newline
Studies have shown that routine-based remembering strategies are good for building habit automaticity. Stawarz et al. \cite{article_dont_forget_your_pill} has produced a set of design requirements for building mobile apps grounded in habit formation theory. These requirements aim to facilitate the formation of reliable routine-based remembering strategies.

\subsubsection*{Chatbots}
Interaction with current habit formation systems is often via a mobile app. This creates a notable difference in the person when the system is removed, as people personalise their phone and it becomes a part of them \cite{article_my_phone_is_part_of_my_soul}. This is also the case with many mobile feedback systems that aid with behaviour change, when the system is removed any improved performance does not persist \cite{article_realtime_feedback_improving_medication_taking, article_dont_kick_habit}.\newline
\newline
Chatbots are a method of communicating with a computer system via a conversation using natural language. They provide a better mobile phone integration for users, as they hook into messaging services users are familiar with. When removing the system, instead of removing an app, users stop messaging a person (the chatbot). This project will build a chatbot to deliver reminders and rewards to users, to test if users sustain habit automaticity.

\subsubsection*{Multi-Modal}
@TODO cite oussamas work here

Studies have shown that good reminder systems should be multi-modal \cite{article_designing_multimodal_reminders_for_home}, providing alternative ways to interact with the user, either visual, auditory or tactile. This increases the likelihood that the delivery method will be pleasant and satisfactory to the user. Incorporating this technique into the chatbot by delivering rewards to users across multiple modalities ensures rewards are intrinsic. But, reminders will be issued on a single mode to limit the scope of this project to test how rewards are effected.

\subsection*{Aims}
This project aims to support habit formation by building a chatbot that delivers reminders and rewards to users. The reminders will be on a single mode and the rewards will be multi-modal.

\subsection*{Objectives}
The chatbot will provide habit tracking with reminder messages as triggers, and rewards as positive reinforcement in three modalities, visual, auditory and tactile.\newline
\newline
The rewards will provide the user with the satisfaction of completing the habit action and encourage them to build user habit automaticity. The visual reward will be a photo, the auditory reward will be an audio clip and the tactile reward will integrate with a wearable to provide tactile feedback from vibrations.

\subsection*{Methodology}
To net the largest amount of users to test, the chatbot will be built using a popular messaging platform, Facebook Messenger.\newline
\newline
A 30-day user study, and a 1-week follow up study, will test the success of the chatbot by evaluating the impact of each modality on facilitating the development of habit automaticity. Chatbot interaction will be removed during the follow up study to test if users continue with the habit. Participants will split into four groups, all groups will receive reminders, three groups will receive rewards each from a different modality, and one group (control group) will not recieve any rewards.\newline
\newline
The user study will gather the following:
\begin{itemize}
  \item quantitative analysis of chatbot interaction
  \item qualitative survey of habit interaction at the end of the study, and end of follow up study @TODO state you are gonna test w survey, rather than `habit interaction'
  \item automaticity test at beginning and end of study
\end{itemize}

\subsection*{Deliverables}
\begin{itemize}
  \item A chatbot that supports habits formation, using notification reminders as triggers and for rewards uses a combination of these 3 rewards
    \begin{itemize}
      \item visual rewards as photos
      \item auditory rewards as audio clips
      \item tactile rewards as vibrations integrated with a wearable
    \end{itemize}
  \item Analysis of the effectiveness of different modalities on [habit automaticity] @TODO better state this / what
  \item Design recommendations for building a habit formation chatbot
\end{itemize}

\subsection*{Added value}
The design recommendations will aid further research into building habit formation systems. The user study will show how effective chatbots are for delivering reminders and rewards and show how different types of reward deliveries through different modalities effect user habit automaticity.\newline
\newline
This research review looks at the literature around habit formation and multi-modal rewards, summarising with design guidelines and a project plan, to test if multi-modal rewards provided by a chatbot support habit formation.

\newpage

