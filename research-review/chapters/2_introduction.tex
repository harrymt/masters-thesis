
\section{Introduction}

\subsection*{Background}
People have goals they want to achieve that require repetitive actions, such as regular exercise or losing weight. Habits can be used to perform these actions with almost no conscious thought in a automatic-like way. Forming a positive habit can increase the chance people achieve these type of goals by changing their behaviour \cite{article_promoting_habit_formation}. There are three elements of habit formation, repetition, contextual cues and positive reinforcement \cite{article_beyond_self_tracking_designing_apps}. Associating the cue with performance and grounding the process with a reward encourages regular repetition, leading to automatic behaviour \cite{article_experiences_of_habit_formation}. Building a new habit requires a contextual cue, to trigger the start of the habit (action), and a reward for positive reinforcement \cite{article_beyond_self_tracking_designing_apps}. For example, when you eat breakfast (trigger), you might write in a journal (action), and then reflect on last week (reward). Studies have shown that the process of creating a new habit takes on average up to 66 days of repetitive use \cite{article_how_habits_formed_modelling_habit_formation}. This is a long time for people to remember, perform and sustain a new habit without any help, if only the majority of people carried around a contexually aware device that could aid us...

\subsection*{Motivation}

\subsubsection*{Habit Formation}
Technology can solve this problem, by reminding and building motivation for repetitive tasks. Mobile phones provide us with an interactive platform that can help with habit formation. Plenty of existing habit-formation systems use apps that guide users through a series of experiences to form a new habit. Although, literature shows these apps are unsuccessful at forming habits because they are not grounded in habit formation theory \cite{article_beyond_self_tracking_designing_apps}. The apps create a dependence on the technology and do not build the automatic reaction to a trigger (habit automaticity) e.g. when people stop using habit forming mobile apps, they also stop performing the habit.\newline
\newline
Studies have shown that routine-based remembering strategies are good for building habit automaticity. \cite{article_dont_forget_your_pill} has produced a set of design requirements for building mobile apps grounded in habit formation theory. These requirements aim to facilitate the formation of reliable routine-based remembering strategies.

\subsubsection*{Chatbots}
Interaction with current habit-formation systems is often via a mobile app. This creates a notable difference in the person when the system is removed, as people personalise their phone and it becomes a part of them \cite{article_my_phone_is_part_of_my_soul}. Removing the habit formation app, has shown to stop people performing the habit.\newline
\newline
Chatbots are a method of communicating with a computer system via a conversation using natural language. They provide a better mobile phone integration for users, as they hook into messaging services users are familiar with. When removing the system, instead of removing an app, users stop messaging a person (the chatbot). This project will build a chatbot to deliver reminders and rewards to users, to test if users sustain habit automaticity.

\subsubsection*{Multi-Modal}
Studies have shown that good reminder systems should be multi-modal \cite{article_designing_multimodal_reminders_for_home}, providing alternative ways to interact with the user, either visual, auditory or tactile. This increases the likelihood that the delivery method will be pleasant and satisfactory to the user. Incorporating this technique into the chatbot by delivering rewards to users across multiple modalities ensures rewards are intrinsic. But, reminders will be issued on a single mode to limit the scope of this project to test test how rewards are effected.

\subsection*{Aims}
This project aims to support habit formation by building a chatbot that delivers reminders and rewards to users. The reminders will be on a single mode and the rewards will be multi-modal.

\subsection*{Objectives}
The chatbot will provide habit tracking with reminder messages as triggers, and rewards as positive reinforcement in three modalities, visual, auditory and tactile.\newline
\newline
The rewards will provide the user with the satisfaction of completing the habit action and encourage them to build user habit automaticity. The visual reward will be a photo, the auditory reward will be a audio clip and the tactile reward will integrate with a wearable to provide tactile feedback.

\subsection*{Methodology}
To net the largest amount of users to test, the chatbot will be built using a popular messaging platform, Facebook Messenger.\newline
\newline
A 30 day user study, and a 1 week follow up study, will test the success of the chatbot by evaluating effectiveness of each modality on habit automaticity. Chatbot interaction will be removed during the follow up study to test if users continue with the habit. Three groups, and a control group, will each receive reminders and rewards from a different modality.\newline
\newline
The user study will gather the following:
\begin{itemize}
  \item quantitative analysis of chatbot interaction
  \item qualitative survey of habit interaction at the end of the study, and end of follow up study
  \item automaticity test at beginning and end of study
\end{itemize}

\subsection*{Deliverables}
\begin{itemize}
  \item A chatbot that supports habits formation, using notification reminders as triggers and for rewards uses a combination of these 3 rewards
    \begin{itemize}
      \item visual rewards as photos
      \item auditory rewards as audio clips
      \item tactile rewards as vibrations integrated with a wearable
    \end{itemize}
  \item Analysis of the effectiveness of different modalities on habit automaticity
  \item Design recommendations for building a habit formation chatbot
\end{itemize}

\subsection*{Added value}
The design recommendations will aid further research into building habit formation systems. The user study will show how effective chatbots are for delivering reminders and rewards and show how different types of reward deliveries through different modalities effect user habit automaticity.\newline
\newline
This research review looks at the literature around multi-modal rewards and habit formation, summarising with design guidelines and a project plan, to test if multi-modal rewards provided by a chatbot support habit formation.

\newpage

