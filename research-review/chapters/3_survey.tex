\newpage

\section{Habit Formation}

To understand how to build an app that supports habit formation, we must discuss how people fundamentally form habits.\newline
\newline
To change a persons behaviour an action needs to be repeated performed to turn into a habit and ensure that the behaviour persists in the future.

\subsection{What are Habits}
Within Psychology, Habits are learned automatic cue-response actions, such actions that will perform automatically in response to another action or trigger that has been actioned repeatedly in the past \cite{article_the_habitual_consumer}.

\subsection{Forming Habits}
Studies have shown people must keep to strict strategies and perform an action repeatedly before it turns into a action that occurs with little concious thought \cite{article_promoting_habit_formation}.

\subsection{Three Elements of Habit Formation}
Forming a habit occurs similarly to how a person changes their behaviour. Research \cite{article_beyond_self_tracking_designing_apps} shows that using these 3 elements ensures an action becomes permanent.

\begin{enumerate}
  \item Repetition
  \item Contextual Cues
  \item Positive Reinforcement
\end{enumerate}

\subsection*{Repetition}
Lally et al. (2010) states that the process of creating a new habit takes on average up to 66 days of repetitive use. The easier the action, the shorter time before the action turns into a habit, from drinking water (18 days), to going to the gym (254 days) \cite{article_how_habits_formed_modelling_habit_formation}. Although repeating an action is not enough to form a habit.

\subsection*{Contextual Cues}
Contextual Cues are actions attached to a context. These act as trigger events to push the person onto performing the action. For example, if you wanted to adopt a stretching habit, you could attach it onto an existing context like brushing your teeth. The contextual cue of brushing your teeth will trigger you to stretch. Literature \cite{article_beyond_self_tracking_designing_apps} shows that attaching habits onto existing event-based cues are easier to remember, when compared with time-based habits, e.g. stretch every 4 hours. These help connect the contextual cue with the habit and builds habit automaticity (CITE: In Beyond self tracking, ref 12.). Further research also shows how multi-cue routines are the most effective. (CITE: Kathys PhD, cant find paper on it)

\subsection*{Positive Reinforcement}
Rewarding a person with positive reinforcement after the action, builds the habit by giving the feeling of satisfaction. Rewards that benefit the person with satisfaction (intrinsic) should be used over monetary gains (extrinsic), due to isseus with extrinsic rewards hindering motivation \cite{article_beyond_self_tracking_designing_apps}.

\subsection*{Automaticity}
- Big up needing technology for this process, 'if only we could do this automatically'

\subsection{Technology}

People find it difficult to change their behaviour. They find it hard to sustain positive habits. Can we use technology to help us?\newline
\newline
There has been little research into how systems can support habit formation and behaviour change. A large number of these systems are mobile apps and are highly rated on app stores. However, most are not grounded in behaviour change theory with research showing habits are not sustained when the app is removed. This is because the apps do not build habit automaticity. Further research shows multi-cue routines develop automaticity and how multiple modalities encourage users to maintain performance \cite{article_realtime_feedback_improving_medication_taking}.\newline
\newline
This project will build upon requirements based on habit-formation theory, presenting a new method of delivering habit-forming techniques to people across multiple modalities using a Chatbot.



- How can we use tech for this problem
Later research discusses concrete strategies for use with mobile technology \cite{article_beyond_self_tracking_designing_apps}.

  \subsubsection*{Current State}
      - Current apps available
      - Kathys Research

  \subsubsection*{Problems}
      \subsubsection*{- Not grounded in science}
        - Kathys research, state
      \subsubsection*{- Lack of automaticity}
        - Doesn't develop when you remove the system
      \subsubsection*{- Dependant ond tech}
        - To do with lack of automaticity
        - However, chatbots are not dependant on tech, as everyone will have a messaging platform, e.g. fb messenger

  \subsection{Requirements}
      \subsubsection*{- Kathys Research}
        - Design requirements
      \subsubsection*{- Taxonomy of...}
        - Reward focused requirements
      \subsubsection*{Finilised requirements}
      Requirements for Habit-Formation systems
      - List of combined requirements
      - However still lack of retention, but, there is hope,
      <
        table of requirements
        - Foucs on just rewards
        - I will use kathys reqs for tracking requirements
      >
      - @TODO multimodal research says using multiple modalities improves retention / automaticity

\newpage

\section{Multi-Modal}
  - What is a Modalitiy
  - Research into how multiple modalities improve retention
  - Different types, most common and @TODO research says these are most effective
  \subsection{Methods of Interaction}
  \subsubsection*{Audio}
      - What is it
      - Why are we choosing it
      - Examples
      \subsubsection*{- Why is it good?}
        - What does it give to us
        - @TODO research
      \subsubsection*{- Reward}
        - Cross check with my requirements
  \subsubsection*{Visual}
      - What is it
      - Why are we choosing it
      - Examples
      \subsubsection*{- Why is it good?}
        - @TODO research
      \subsubsection*{- Reward}
        - Cross check with my requirements
  \subsubsection*{Tactile}
      - What is it
      - Why are we chosing it
      - Examples
      - Wearables, Fitbit
    \subsubsection*{Why is it good?}
        - @TODO research
    \subsubsection*{Reward}
        - Cross check with requirements

  \subsection{Delivering Rewards}
    - Table of multimodal reward implementation strategies
    - Table of requirements matched with modalities
    - If you could implement this, you could increase users automaticity for habits

\newpage
\section{Chatbot Design}
  \subsection{Design Considerations}
    - Setup:
      - Setup the bot via a messaging platform, such as fb messenger
    - Trigger:
        - Either A, certain configured time of the day
        -        B: No trigger
        -        C: Around a specific time
    - Action:
      - Choose habit from list of habits
      - Perform
      - Use app to track the action
    - Reward:
      - You get one of these rewards, based on modalitiy selected
      - Vision
        - Through message, of an image or gif
        - Could be: App, or message, gif
      - Audatory
        - Through phone via bot, link to mp3/spotify/apple music
        - Could be: App
      - Tactic
        - Through wearable
        - Could be: App, bot triggers wearbale alarm



  \subsection{User Flow}
      - Pre-Start
        - Choose daily habit type from list of X, e.g. 1 press up before breakfast
        - Enable notifications or fitbit if chosen
        - Time action / reward, variable rewards, e.g. then work out average time to send, or none
      - Start:
        - New day
        - @ trigger time, send reminder, if set, notification
        - Open notification, do habit, press tracked
        - Get reward type

% \newpage
% TODO maybe move this into design
%   \subsection{Implementation}
%     - Web app could be chosen because its easiest and achievable, however the ease of use with a chatbot, integrated into fb messenger means everyone can use it on multiple devices. The addition, means that people get used to the UI.
%   \subsubsection{Technology}
%       - 3 Types:
%         1. specific apps
%           - Bad cuz takes a long time
%         2. Web Apps
%           - Still another app
%         3. Chatbot
%           - Good useful
%       - Android/iOS/chatbot specific notifications from web app
%       - Save to home screen
%   \subsubsection{Implementing Rewards}
%       - Vision
%         - Send notification
%         - Show nice visuals
%       - Audio
%         - Send notification
%         - Play uplifting music
%       - Tactic
%         - A.P.I. sets wearable alarm
%         - Wearable (fitbit) issues and tracks alarm times
%     \subsection{Components}
%       [app] -------> (Database) -----> at certain time ---> Send notification to trigger type of reward
%       [ big button that says track]
%       taskname textbox

% \newpage
% \section*{Implementation old}

% Habit formation, Don't Kick the Habit \cite{article_dont_kick_habit}, stopping the behaviour when one stops using behaviour change apps.
% Measuring Habit strength \cite{article_habit_strength} \cite{article_habit_measurement}.

% Multi-Modal interaction, using multiple modalities for habit formation \cite{}, using wearable devices, such as a Fitbit device is plausible for a study \cite{article_wearable_good}.

% Gamification elements \cite{f2p_games_how_to}, can we use any elements from Free-To-Play games?


