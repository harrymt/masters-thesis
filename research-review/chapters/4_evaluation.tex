
\newpage
\section{Evaluation}

Literature demonstrates the importance of designing for health. {Stawarz, K. et al. presents three questions for evaluating technologies for behaviour change \cite{article_designing_for_health_behaviour_change_hci}.
\begin{enumerate}
\item Is it usable?
\item Does it meet users needs?
\item Is it effective?
\end{enumerate}
These questions are drawn from several fields of study, not just HCI research, considering long-term effects and efficiency. These will serve as the basis for evaluating this project.

\subsection{User studies}
The number of repetitive days required for an action to be considered a habit varies based on the complexity of the action \cite{article_how_habits_formed_modelling_habit_formation}. Simple actions, such as drinking 2 glasses of water a day, can a minimum of 18 days to form. The actions used for this project shall be consider as simple, e.g. stretching for 30 seconds. Therefore, a user study above 18 days will be  enough time for a simple action to form into a habit.\newline
\newline
A 30 day study will test the success of the chatbot by evaluating effectiveness of each modality on habit automaticity. Chatbot interaction will be removed during the follow up study to test if users continue with the habit. Three groups, and a control group, will each receive reminders and rewards from a different modality. The above three questions will be used as the starting point for surveys that ask for feedback about the implementation. Habit strength will be measured using a 12 question questionnaire by Verplanken et al. 2003 \cite{article_habit_strength}. This test will occur before the study, after the study and after the follow up study.

\subsection{Testing automaticity}
Finally, users habit behavioural automaticity index will be measured, by Lally et al. 2012. \cite{article_habit_measurement} to test the impact each modality has on habit automaticity. This will test the hypothesis.  Users habit automaticity will be tested before the study, after the study and after the 1 week follow up study.
