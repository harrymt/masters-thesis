
\newpage
\section{Evaluation}

Literature demonstrates the importance of designing for health @TODO find multiple cites about importance for designing for health. Stawarz et al. \cite{article_designing_for_health_behaviour_change_hci} and @TODO Find primary source for \cite{article_designing_for_health_behaviour_change_hci} presents three questions for evaluating technologies for behaviour change.
\begin{enumerate}
\item Is it usable?
\item Does it meet users needs?
\item Is it effective?
\end{enumerate}
These questions are drawn from several fields of study, not just HCI research, considering long-term effects and efficiency. These will serve as the basis for evaluating this project.

\subsection{User Study}
The number of repetitive days required for an action to be considered a habit varies based on the complexity of the action \cite{article_how_habits_formed_modelling_habit_formation}. Simple actions, such as drinking 2 glasses of water a day, can a minimum of 18 days to form. The actions used for this project shall be consider as simple, e.g. stretching for 30 seconds. Previous habit formation research \cite{article_beyond_self_tracking_designing_apps} showed an increase in habit automaticity after a 4-week study, therefore this project will conduct a 4-week study to test users habit automaticity.\newline
\newline
A 30 day study will test the success of the chatbot by evaluating effectiveness of each modality on habit automaticity. Chatbot interaction will be removed during the follow up study to test if users continue with the habit. Three groups, and a control group, will each receive reminders and rewards from a different modality. The above three questions will be used as the starting point for surveys that ask for feedback about the implementation.

\subsection{Testing Habit Strength}
 Habit strength will be measured using a validated 12-question habit strength questionnaire that specifically looks at automaticity \cite{article_habit_strength}. Automaticity will also be measured using a validated subset of the questionaire from \cite{article_habit_strength} to test users habit behavioural automaticity index \cite{article_habit_measurement}. This will show the impact each modality has on habit automaticity and evaluate the hypothesis. These tests \cite{article_habit_strength, article_habit_measurement} will occur half-way through study (at 2 weeks), after the study and after the follow up study.