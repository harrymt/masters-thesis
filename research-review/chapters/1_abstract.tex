
\section*{Executive Summary}
% \section*{Abstract}
Testing lallys study results again with positive reinforcement / rewards, how does this affect things after 66 days? Different types of rewards.

Building a Habit Formation app that focuses on Rewards. Current Habit Formation apps are not grounded in habit formation/behaviour change theory, so when people stop using them, they don't continue with the habit. [Reward theory is mainly based on a single modality, ] Using Kathys research I will test several theories [that use multiple modalities] in an implementation to test their effectiveness. Resulting in an app that assists you with habit formation, for 18 to 254 days to create the habit, and has the ability to give you different type of rewards based on different theories. When the app is removed, you will continue to do the habit. I will run [x] studies for (18 to 254) + 7 days, to test the [x] reward system implementations, and see if the habit is continued for a week after.

% Habits are actions we do with almost no conscious thought. Building a positive habit requires a methodical process, from a trigger to a reward. For example, when you eat breakfast, you might write in a journal about your life. Daily actions are often easier to sustain than longer ones, although the process of creating a habit takes up to 66 days of repetitive use. Technology can change peoples behaviour and encourage them to form a habit by guiding users through a series of experiences. Trigger, Action and Reward. Theories have explored different methods of providing people with rewards, however, literature shows that these implementations are never as affective as the theory states.

% This project focuses on habit reward systems and how theoretical methods of sustaining habits are put into practice. This project aims to implement different types of reward deliveries through different modalities and also provides a set of design recommendation for building habit forming systems.


% TODO: Delete
% Systems to assist with habit adoption are intended to begin new behaviours and try to maintain the same behaviour even when users disengage with the system. Prior research evaluates habit adoption with single modality constructs to enforce habit behaviour, like visual cues or vibrating alarm reminders. However, prior research has established a dependency between on-going habit adoption system use and lasting change. In this paper, we propose and evaluate three systems that use different modality configurations for habit adoption. The is for users to keep engaged with the habit adoption when the system is removed. A 30 day user study will be conducted to evaluate the use of each system and test how users compliance with conducting the habit improves when the system is in place and then again when the system is removed. We conclude with an implementation and evaluation of the three proposed systems for habit adoption and present a set of implications for the design of a system in this domain.