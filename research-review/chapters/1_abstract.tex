\section*{Executive Summary}

Habits are automatic actions that require little effort. For example the action of washing your hands automatically after using the toilet is a habit. Forming new positive actions can lead to many health benefits and require three elements to form the action into a habit: Repetition, contextual cues and positive reinforcement. Using these three elements, on average, will take 66 days of repetitive use to become a habit.\newline
\newline
Mobile technology can help form habits with apps using these three elements, by reminding us to repeat an action, helping us build contextual cues and rewarding with positive reinforcement. However most existing mobile apps are not grounded in theory and build repetitive actions rather than habit automaticity. People become dependent on the app rather than the habit and when the app is eventually removed, the habit performance decreases. Building habit automaticity is difficult, but motivation has been shown to the key building factor. Rewards after an action builds motivation to complete the action, by giving people positive reinforcement granting them satisfaction from completing the action. This is a key factor in habit formation, strengthening the habit and building motivation and automaticity. This project will focus on this aspect of habit formation to experiment with the impact rewards have on habit formation.\newline
\newline
Good habit forming systems should also use different methods of interacting with users from different modalities, giving users a choice of interaction. This project will combine this insight with rewards, aiming to focus on investigate the effect rewards from different modalities have on habit formation.\newline
\newline
This project aims to deliver these techniques using a chatbot instead of an app. Chatbots are a method of communicating with a computer system using natural language. They provide deeper integration into users mobile phone, as they hook into messaging services users are familiar with. This presents a novel method of delivering rewards to people and insight into the impact of using a chatbot to form habits.\newline
\newline
The literature presents us with a set of design requirements for habit forming apps that build habit automaticity, focused on routine-based remembering strategies. Combined with another study, showing that good reminder and remembering systems should also interact with the user across different modalities, to allow delivery of triggers and rewards across a modality to suit different types of users. This project will base the design of the chatbot on this set of requirements and principles to deliver habit rewards.\newline
\newline
This project will evaluate the chatbot during a 30-day controlled test. Participants will engage daily with the bot for 30 days, then for a further 7 days, all interaction with the bot will be stopped to test if users continue to perform their chosen habit after the bot is removed. The chatbot will provide habit tracking using reminders as triggers, and rewards from three modalities, visual, auditory and tactile. The user study will test the success of the chatbot by evaluating the impact of each reward from a different modality and how this impacts habit automaticity. Participants shall split into four groups, a control group shall receive 0 rewards and the remaining three groups will each receive rewards from a different modality.\newline
\newline
The project will deliver insight into how rewards from different modalities effect habit strength and opens up new research avenues for the investigating the use of chatbots as vehicles for promoting behaviour change.

\newpage

\section*{Definitions}

\textbf{Human-Computer Interaction (HCI)} - Field of computer science that studies how people interact with computers.\newline
\newline
\textbf{Modality} - In the context of HCI, a modality or mode is the classification of a single independent channel of sensory input or output between a computer and a human.\newline
\newline
\textbf{Multi-Modal Interaction} - Provides the user with multiple modalities or modes for interacting with a system.\newline
\newline
\textbf{Chatbot} - A method of communicating with a computer system via a conversation using natural language.

\newpage
