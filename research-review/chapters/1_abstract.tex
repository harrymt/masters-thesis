
\section{Executive Summary}

Habits are automatic actions that require little effort. Such as automatically washing your hands, after using the toilet. Forming new positive habits gives us lots of benefits, in health and other areas. Forming a new habit requires 3 elements. Repetition, contextual cues and positive reinforcement. On average habits takes 66 days of repetitive use to become automatic.\newline
\newline
Mobile technology can help form habits, by reminding use to repeat the habit, giving us contextual cues and rewarding us to form positive reinforcement. But, most existing systems are not grounded in theory and build repetitive actions rather than habit automaticity. People then become dependent on the technology, rather than the habit. This is bad because when we remove the system, the habit stops. Therefore, successful habit forming systems need to build habit automaticity.\newline
\newline
The current state of habit-forming mobile systems use apps for interaction, encouraging the user to repeat tasks using the app. This creates a dependancy between the app and the user. Where the users are dependaent on the app to continue to repeat the habit. When the app is removed, the user stops repeating the habit as they are dependent on the app. This has been shown to decrease habit automaticity.\newline
\newline
This project aims to explore a different method of forming habits, by using a chatbot.\newline
Chatbots are a method of communicating with a computer system using natural language. They provide a better mobile phone integration for users, as they hook into messaging services users are familiar with, and when the system is removed, instead of removing an app, the users stops having a conversation with the chatbot.\newline
\newline
The literature presents us with a set of design requirements for habit forming apps that build habit automaticity, focused on routine-based remembering strategies. Combining this with another study, showing that good reminder and remembering systems should also interact with the user across different modalities. To allow delivery of triggers and rewards across a modality to suit different types of users.\newline
This project will base the design of the chatbot on these requirements and principles to deliver reminders and rewards across multiple modalities.\newline
\newline
This project aims to build a chatbot that supports habit formation, by building habit automaticity with multi-modal interaction over 66 days. After removing the chatbot, users should continue to perform the habit. The chatbot will provide habit tracking by means of reminders as a trigger, and rewards in three modalities, visual, auditory and haptic. A 66 day user study, and a 1 week follow up study, will test the success of the chatbot by evaluating effectiveness of each modality on habit automaticity. Chatbot interaction will be removed during the follow up study to test if users continue with the habit. Three groups, and a control group, will each receive reminders and rewards from a different modality.\newline
\newline
The project will deliver a chatbot, design recommendations and analysis of the effectiveness of different modalities on habit automaticity.

\newpage

\section*{Definitions}

\textbf{Human-Computer Interaction (HCI)} - Field of computer science that studies how people interact with computers.\newline
\newline
\textbf{Modality} - In the context of HCI, a modality or mode is the classification of a single independent channel of sensory input or output between a computer and a human.\newline
\newline
\textbf{Multi-Modal Interaction} - Provides the user with multiple modalities or modes for interacting with a system.\newline
\newline
\textbf{Chatbot} - A method of communicating with a computer system via a conversation using natural language.

\newpage
