
\section{Executive Summary}

Habits are automatic actions that require little effort. Forming new positive habits gives us lots of benefits, such as washing our hands after using the toilet. Forming a new habit requires repetition, contextual cues and positive reinforcement and on average takes 66 days to become automatic. Mobile technology can help form habits, by reminding and rewarding us. But, most existing systems are not grounded in theory and build repetitive actions rather than habit automaticity. People then become dependent on the technology, rather than the habit. This is bad because when we remove the system, the habit stops. Therefore, successful habit forming systems need to build habit automaticity.\newline
\newline
Interaction with these systems is often with a mobile app which creates a mental link with the app and the user. When removing an app, the mental link is also removed, which decreases habit automaticity. This project aims to explore a different method of forming habits, with a chatbot. Chatbots are a method of communicating with a computer system using natural language. They provide a better mobile phone integration for users, as they hook into messaging services users are familiar with. When removing the system, instead of removing an app, the users stops having a conversation with the chatbot.\newline
\newline
The literature presents us with design requirements for habit forming apps that build habit automaticity, focused on routine-based remembering strategies. Studies have also shown that good reminder/remembering systems should also interact with the user across different modalities. This would allow delivery of triggers and rewards across a modality to suit different types of users. This project will base the design of the chatbot on these requirements delivering reminders and rewards across multiple modalities.\newline
\newline
This project aims to build a successful habit formation chatbot, one which should build habit automaticity via multi-modal interaction over 66 days. After removing the chatbot users should continue to perform the habit. The chatbot will provide habit tracking with reminder messages as triggers, and rewards in three modalities, visual, auditory and haptic. A 66 day user study, and a 1 week follow up study, will test the success of the chatbot by evaluating effectiveness of each modality on habit automaticity. Chatbot interaction will be removed during the follow up study to test if users continue with the habit. Three groups, and a control group, will each receive reminders and rewards from a different modality.\newline
\newline
The project will deliver a chatbot, design recommendations for building habit formation systems and analysis of the effectiveness of different modalities on habit automaticity.

\newpage

\section*{Definitions}

\textbf{Human-Computer Interaction (HCI)} - Field of computer science that studies how people interact with computers.\newline
\newline
\textbf{Modality} - In the context of Human-Computer interaction, a mode is the classification of a single independent channel of sensory input/output between a computer and a human.\newline
\newline
\textbf{Multi-Modal Interaction} - Provides the user with multiple modes for interacting with a system.\newline
\newline
\textbf{Chatbot} - A method of communicating with a computer system via a conversation using natural language.

\newpage
