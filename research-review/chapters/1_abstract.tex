
\section*{Abstract}

Habits are actions we do with almost no conscious thought. Building a positive habit requires a methodical process, from a trigger to a reward. For example, when you eat breakfast, you might write in a journal about your life. Daily actions are often easier to sustain than longer ones, although the process of creating a habit takes up to 66 days of repetitive use. Technology can change peoples behaviour and encourage them to form a habit by guiding users through a series of experiences. Trigger, Action and Reward. Theories have explored different methods of providing people with rewards, however, literature shows that these implementations are never as affective as the theory states.

This project focuses on habit reward systems and how theoretical methods of sustaining habits are put into practice. This project aims to implement different types of reward deliveries through different modalities and also provides a set of design recommendation for building habit forming systems.


TODO: Delete
Systems to assist with habit adoption are intended to begin new behaviours and try to maintain the same behaviour even when users disengage with the system. Prior research evaluates habit adoption with single modality constructs to enforce habit behaviour, like visual cues or vibrating alarm reminders. However, prior research has established a dependency between on-going habit adoption system use and lasting change. In this paper, we propose and evaluate three systems that use different modality configurations for habit adoption. The is for users to keep engaged with the habit adoption when the system is removed. A 30 day user study will be conducted to evaluate the use of each system and test how users compliance with conducting the habit improves when the system is in place and then again when the system is removed. We conclude with an implementation and evaluation of the three proposed systems for habit adoption and present a set of implications for the design of a system in this domain.