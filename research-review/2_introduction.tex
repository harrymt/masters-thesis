

\section{Executive Summary}
Systems to assist with habit adoption are intended to start new behaviours and try to maintain the same behaviour even when users disengage with the system. Existing systems aimed at improving habit adoption use single modality constructs to enforce habit behaviour, like visual cues or vibrating alarm reminders. However, prior research has established a dependency between on-going habit adoption system use and lasting change. In this paper, we propose and evaluate three systems that use different modality configurations for habit adoption in conjunction with gamification. The aim is for users to keep engaged with the habit adoption when the system is removed. A 30 day user study will be conducted to evaluate the use of each system and test how users compliance with conducting the habit improves when the system is in place and then again when the system is removed. We conclude with an implementation and evaluation of the three proposed systems for habit adoption and present a set of implications for the design of a system in this domain.

\subsubsection*{Motivation}
Maintaining habit behaviour when the system is removed is difficult! We try to counter this and we can feasibly find a way for people to adopt a habit without relying on a system. 


\subsubsection*{Project Type}
In this paper, we propose and evaluate three configurations for habit adoption.
The proposed project is split 40/60 between researching different configurations and implementing three configurations.


\subsubsection*{Aims and Objectives}
The aim is to achieve habit adoption when a configuration is removed.


\subsubsection*{Methodology}
Prior research into different habit adoption systems will refine the design requirements for three system configurations. System A, B and C.\newline
Will use design requirements from \cite{article_dont_forget_your_pill}.
These systems will be constructed using a staggered approach to optimise the use of time.
After testing of each system configuration, each individual system will be evaluated and the results will be compared to produce a set of implications for the design of a system.


\subsubsection*{Deliverables}
Three evaluated systems to adopt habits. Design requirements from the results of testing these three systems.


\subsubsection*{Added Value}
Results of studies and design requirements.


